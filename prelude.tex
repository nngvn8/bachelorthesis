
%%%%%%%%%%%%%%%%%%%%%%%%%%%%%%%%%%%% PACKAGES %%%%%%%%%%%%%%%%%%%%%%%%%%%%%%%%%%%%%%%%%%

\usepackage{inputenc,fontenc}
\usepackage[a4paper,margin=3cm]{geometry}
\usepackage[english]{babel}
%\usepackage[german]{babel}
%\usepackage[fixlanguage]{babelbib}


\usepackage{bbold}
\usepackage{amsthm}
\usepackage{amsmath}
\usepackage{amssymb} % doteqdot
\usepackage[dvipsnames]{xcolor}
\usepackage{standalone}
\usepackage{tikz}[mode=buildnew]
\usepackage{cite}
\usepackage{xspace}
\usepackage{relsize}
\usepackage{mathtools} % mathclap
%\usepackage{MnSymbol}
\usepackage{hyperref}
\usepackage{url}
\usepackage{listings} % for code
\usepackage[T1]{fontenc} %<
\hypersetup{
	colorlinks,
	citecolor=black,
	filecolor=black,
	linkcolor=black,
	urlcolor=black
}
\usepackage{pgfplots}
\pgfplotsset{compat=1.18}
%\usepackage{courier} %% Sets font for listing as Courier. But also for url and texttt!
\usepackage{listings, xcolor}
\usepackage{graphicx}
\usepackage{subcaption}

\usepackage[section]{placeins} % make figures not float intoother subsection
\let\Oldsection\section
\renewcommand{\section}{\FloatBarrier\Oldsection}
\let\Oldsubsection\subsection
\renewcommand{\subsection}{\FloatBarrier\Oldsubsection}

\usetikzlibrary{calc}
%\usepackage{xparse} % \newDocumentCommand for multiple optional arguments
%\usepackage{titlecaps}

%\captionsetup[lstlisting]{name=Code Snippet}

\lstdefinelanguage{prism}{
	sensitive=true,
	morecomment=[l]{//},
	morecomment=[s]{/*}{*/},
	morestring=[b]",
	morekeywords={
		auto, break, case, char, const, continue, default, do, double, else,
		enum, extern, float, for, goto, if, int, long, register, return, short,
		signed, sizeof, static, struct, switch, typedef, union, unsigned, void,
		volatile, while
	},
	keywordstyle=\color{RoyalBlue}\bfseries,
	commentstyle=\color{gray}\itshape,
	stringstyle=\color{purple},
	numberstyle=\tiny\color{gray},
	identifierstyle=\color{black},
	basicstyle=\ttfamily\small,
	showstringspaces=false,
	%	breaklines=true, % avoids closing parenthesis from being colored
	frame=lines,
	framesep=5pt,	
	%	backgroundcolor=\color{gray!10},
	tabsize=4,
	%	numbers=left,
	captionpos=b,
	literate={
		{[}{{\textcolor{CarnationPink}{[}}}1
		{]}{{\textcolor{CarnationPink}{]}}}1
		{)}{{\textcolor{CarnationPink}{)}}}1
		{(}{{\textcolor{CarnationPink}{(}}}1
		{=}{{\textcolor{CarnationPink}{=}}}1
		{*}{{\textcolor{CarnationPink}{*}}}1
		{+}{{\textcolor{CarnationPink}{+}}}1
		{;}{{\textcolor{CarnationPink}{;}}}1
		{.}{{\textcolor{CarnationPink}{.}}}1
		{->}{{\textcolor{CarnationPink}{->}}}2
		{0}{{\textcolor{Tan}{0}}}1
		{1}{{\textcolor{Tan}{1}}}1
		{2}{{\textcolor{Tan}{2}}}1
		{3}{{\textcolor{Tan}{3}}}1
		{4}{{\textcolor{Tan}{4}}}1
		{5}{{\textcolor{Tan}{5}}}1
		{6}{{\textcolor{Tan}{6}}}1
		{7}{{\textcolor{Tan}{7}}}1
		{8}{{\textcolor{Tan}{8}}}1
		{9}{{\textcolor{Tan}{9}}}1
		{re1}{{\textcolor{black}{\hspace{-3pt}re1}}}3
		{re2}{{\textcolor{black}{\hspace{-3pt}re2}}}3
		{ig1}{{\textcolor{black}{\hspace{-3pt}ig1}}}3
		{ig2}{{\textcolor{black}{\hspace{-3pt}ig2}}}3
		{M1}{{\textcolor{black}{\hspace{-3pt}M1}}}2
		{M2}{{\textcolor{black}{\hspace{-3pt}M2}}}2
		{p1}{{\textcolor{black}{\hspace{-3pt}p1}}}2
		{p2}{{\textcolor{black}{\hspace{-3pt}p2}}}2
		{p3}{{\textcolor{black}{\hspace{-3pt}p3}}}2
		{(got}{{\textcolor{gray}{(got\hspace{-4pt}}}}4
		{left)}{{\textcolor{gray}{left)}}}5
		{right)}{{\textcolor{gray}{right)}}}6
		{forks)}{{\textcolor{gray}{forks)}}}6		
%		{(got right)}{{\textcolor{gray}{\hspace{-5pt}(got right)}}}11
%		{(got forks)}{{\textcolor{gray}{\hspace{-5pt}(got forks)}}}11
%		{t)}{{\textcolor{gray}{\hspace{-20pt}t)}}}4
		{true}{{\textbf{true}}}4
		%		{c_max_time}{{\textcolor{Tan}{c_max_time}}}10			
	}
}


\newcommand{\commentcolor}{\color{green!60!black}}
\newcommand{\commentcolorize}[1]{{\commentcolor#1}}

\lstdefinestyle{javaStyle}{
	language=Java,
	basicstyle=\smaller\ttfamily,
	keywordstyle=\color{blue}\bfseries,
	commentstyle=\commentcolor,
	stringstyle=\color{purple},
	%	numbers=left,
	%	numberstyle=\tiny\color{gray},
	%	stepnumber=1,
	%	numbersep=8pt,
	%	backgroundcolor=\color{gray!10},
	showspaces=false,
	showstringspaces=false,
	showtabs=false,
	frame=lines,
	tabsize=4,
	captionpos=b,
	breaklines=true,
	breakatwhitespace=false,
	%	escapeinside={\%*}{*)}
	escapechar=\~
}


%%%%%%%%%%%%%%%%%%%%%%%%%%%%%%%%%%%% THEOREMSTYLES %%%%%%%%%%%%%%%%%%%%%%%%%%%%%%%%%%

\theoremstyle{definition}
\newtheorem{definition}{Definition}[section]
\newtheorem{exmp}{Example}[section]
%\AfterEndEnvironment{definition}{\noindent\ignorespaces}

\theoremstyle{theorem}
\newtheorem{theorem}{Satz}[section]
\newtheorem{proposition}{Proposition}[section]
%\AfterEndEnvironment{theorem}{\noindent\ignorespaces}

\theoremstyle{korollary}
\newtheorem{korollary}{Korollar}[section]
%\AfterEndEnvironment{korollary}{\noindent\ignorespaces}


\tikzset{
	mstate/.style={draw, circle, minimum size=.94cm}, 
	gstate/.style={draw, rectangle, minimum size=.8cm},
	varstate/.style={draw,rectangle, rounded corners, minimum size=1}, 
	trans/.style={->, thick},
	bendtrans/.style={->, thick, bend left=10},
	bendtransr/.style={->, thick, bend right=10},
	init/.style={initial, initial distance=6pt, initial text=},
	every loop/.style={min distance=5pt, looseness=8},
	>=latex
}
\usetikzlibrary{automata,positioning}

%auto shift/.style={auto=right,->,
%	to path={ let \p1=(\tikztostart),\p2=(\tikztotarget),
%		\n1={atan2(\y2-\y1,\x2-\x1)},\n2={\n1+180}
%		in ($(\tikztostart.{\n1})!1mm!270:(\tikztotarget.{\n2})$) -- 
%		($(\tikztotarget.{\n2})!1mm!90:(\tikztostart.{\n1})$) \tikztonodes}},

%%%%%%%%%%%%%%%%%%%%%%%%%%%%%%%%%%% MY MACROS %%%%%%%%%%%%%%%%%%%%%%%%%%%%%%%%%%%%%%%%%
%formatting
\newcommand{\comment}[2]{{\color{#1}#2}}
\newcommand{\redcomment}[1]{{\color{red}#1}}
\newcommand{\purpcomment}[1]{{\color{pink}#1}}
\newcommand{\bluecomment}[1]{{\color{blue}#1}}
\newcommand{\mt}[1]{\ensuremath{{#1}}\xspace}
\newcommand{\mynewcommand}[2]{\newcommand{#1}{\mt{#2}}} %% currently not used becaue of ide highlighting
\newcommand{\arr}{\mt{\to}}

%model checking terms
\newcommand{\mimicrel}{\mt{\mathcal{R}}}
\newcommand{\bisimeq}{\mt{\;\!\sim\;\!}}
\newcommand{\simorder}{\mt{\;\!\preceq\;\!}}
\newcommand{\simequiv}{\mt{\;\!\simeq\;\!}} %command already defined
\newcommand{\relts}{\mt{\;\!\bullet_{_{\tiny{TS}}}\;\!}}
\newcommand{\rel}{\mt{\;\!\bullet\;\!}}

%own names
\newcommand{\nm}[1]{#1\xspace}
\newcommand{\mdpN}{\nm{MDP}}
\newcommand{\mdpsN}{\nm{MDPs}}
\newcommand{\viewN}{\nm{view}}
\newcommand{\viewNC}{\nm{View}}
\newcommand{\viewsN}{\nm{views}}
\newcommand{\viewsNC}{\nm{Views}}
\newcommand{\grpfctsubN}{\nm{detached grouping function}}
\newcommand{\grpfctsubNC}{\nm{detached grouping function}}
\newcommand{\grpfctsubNCC}{\nm{Detached Grouping Function}}
\newcommand{\grpfctN}{\nm{grouping function}}
\newcommand{\grpfctNC}{\nm{Grouping function}}
\newcommand{\grpfctNCC}{\nm{Grouping Function}}
\newcommand{\grpfctsN}{\nm{grouping functions}}
\newcommand{\grpfctsNC}{\nm{Grouping functions}}
\newcommand{\grpfctsNCC}{\nm{Grouping Functions}}
\newcommand{\stmimicN}{\nm{state-mimic}}
\newcommand{\stmimicsN}{\nm{state-mimics}}
\newcommand{\stmimickingN}{\nm{state-mimicking}}
\newcommand{\stmimickedN}{\nm{state-mimicked}}
%\newcommand{\chosenphtypeNCC}{\nm{Transition System}}
%\newcommand{\chgphNC}{\nm{Transition system}}
%\newcommand{\chgphN}{\nm{transition system}}
%\newcommand{\chgphsNCC}{\nm{Transition Systems}}
%\newcommand{\chgphsNC}{\nm{Transition systems}}
%\newcommand{\chgphsN}{\nm{transition systems}}
\newcommand{\chgphNCC}{\nm{MDP}}
\newcommand{\chgphNC}{\nm{MDP}}
\newcommand{\chgphN}{\nm{MDP}}
\newcommand{\achgphN}{\nm{an MDP}}
\newcommand{\chgphsNCC}{\nm{MDPs}}
\newcommand{\chgphsNC}{\nm{MDPs}}
\newcommand{\chgphsN}{\nm{MDPs}}
\newcommand{\parllcompN}{\nm{parallel composition}}
\newcommand{\parllcompNC}{\nm{Parallel composition}}
\newcommand{\parllcompNCC}{\nm{Parallel Composition}}
\newcommand{\parllcompsN}{\nm{parallel compositions}}
\newcommand{\parllcompsNC}{\nm{Parallel compositions}}
\newcommand{\parllcompsNCC}{\nm{Parallel Compositions}}
\newcommand{\sccN}{\nm{SCC}}
\newcommand{\sccsN}{\nm{SCCs}}
\newcommand{\bsccN}{\nm{BSCC}}
\newcommand{\bsccsN}{\nm{BSCCs}}
\newcommand{\jgrapht}{\nm{jGraphtT}}
\newcommand{\prism}{\nm{{PRISM}}}

\newcommand{\outactident}{\nm{OutActionsIdent}}

%names
\newcommand{\iffN}{\nm{if and only if}}
\newcommand{\tsN}{\nm{TS}}

%% outactions identical
\newcommand{\outactidentstrong}{\nm{strong}}
\newcommand{\outactidentweak}{\nm{weak}}

% CORE DEFINITIONS
\newcommand{\grpfct}[1][\viewppty]{\mt{F_{#1}}}
\newcommand{\grpfctsub}[1][\viewppty]{\mt{\tilde{F}_{#1}}}
%\newcommand{\grpfctimg}[1]{\mt{{\grpfct}[{#1}]}}
%\newcommand{\fctimg}[2]{\mt{{#1}[{#2}]}}
\newcommand{\eqrelview}{\mt{R}}
\newcommand{\eqclassv}[1][\state]{\mt{\eqclass{#1}{\eqrelview}}}
\newcommand{\eqclasssetv}[1][\states]{\mt{{#1}/\eqrelview}} %OLD: \bigcup_{\state \in \states} \eqclassv
\newcommand{\viewid}{\mt{\mdp}}
\newcommand{\view}[1][\viewppty]{\mt{\viewid_{#1}}}
\newcommand{\imggrp}{\mt{\arbset}}
\newcommand{\imggrpsub}{\mt{X}}
\newcommand{\viewppty}{\mt{\theta}}
\newcommand{\pll}{\mt{\;\!\pllpure\;\!}}
\newcommand{\pllrev}{\mt{\pllpure^{-1}}}
\newcommand{\pllpure}{\mt{||}}
\newcommand{\compselectset}{\mt{Z}}
\newcommand{\compselectpure}{\mt{\pllpure_\compselectset}}
\newcommand{\compselect}{\mt{\;\pllpure_\compselectset\;}}
\newcommand{\remstates}{\mt{\bigcup_{\state \in \states \setminus \states_1}\{\{\state\}\}}}
\newcommand{\nogroupstates}[1][\states_2]{\mt{\bigcup_{\state \in \states \setminus {#1}}\{\{\state\}\}}}
\newcommand{\remelem}{\mt{\bullet}}
\newcommand{\nogroupset}{\mt{\xi}}
\newcommand{\remset}{\mt{\{\remelem\}}}
\newcommand{\gfctpll}{\mt{\grpfct[\pll]}}
\newcommand{\group}{\mt{\top}}
\newcommand{\imggrpbinview}{\mt{\{\remelem, \notppty\}}}
\newcommand{\viewappset}{\mt{\tilde{\states}}}
\newcommand{\hasppty}{\mt{\top}}
\newcommand{\notppty}{\mt{\bot}}
\newcommand{\disregardelem}{\mt{\Delta}}
\newcommand{\disregardelements}{\mt{{\disregardelem_1, \dots, \disregardelem_n}}}


% EXAMPLE VIEWS
\newcommand{\pptyatomicprops}{\mt{\atomicprops}}
\newcommand{\pptyinitstates}{\mt{\initstates}}
\newcommand{\pptyinactsetsize}{\mt{|\inacts(\state)|}}
\newcommand{\pptyhasoutact}{\mt{\exists\outact}}
\newcommand{\pptyminoutact}[2]{\mt{#1\leq#2}}
\newcommand{\pptymaxoutact}[2]{\mt{#2\leq#1}}
\newcommand{\pptyspanoutact}[3]{\mt{#1\leq#2\leq#3}}
\newcommand{\pptyoutactsetsize}{\mt{|\outacts(\state)|}}
\newcommand{\pptyoutactsingle}{\mt{|\outacts(\state)|_1}}
\newcommand{\pptystrongoutactident}{\mt{\outacts(\state)_=}}
\newcommand{\pptyweakoutactident}{\mt{\outacts(\state)_\approx}}
\newcommand{\pptyhasinact}{\mt{\exists\inact}}
\newcommand{\pptymininact}[2]{\mt{#1\leq#2}}
\newcommand{\pptymaxinact}[2]{\mt{#2\leq#1}}
\newcommand{\pptyspaninact}[3]{\mt{#1\leq#2\leq#3}}
\newcommand{\pptyinactsingle}{\mt{|\inacts(\state)|_1}}
\newcommand{\pptystronginactident}{\mt{\inacts(\state)_=}}
\newcommand{\pptyweakinactident}{\mt{\inacts(\state)_\approx}}
\newcommand{\pptyparamvalueseq}{\mt{\var = \varval}}
\newcommand{\pptyparamvaluesneq}{\mt{\var \neq \varval}}
\newcommand{\pptyparamdnf}{\mt{VarDNF}}
\newcommand{\pptyparamcnf}{\mt{VarCNF}}
\newcommand{\pptyparamvalueseqopt}{\mt{\var = \varval}}
\newcommand{\pptyparamvalident}{\mt{Var:\varval}}
\newcommand{\pptydistance}{\mt{\overrightarrow\distpath}}
\newcommand{\pptydistancerev}{\mt{\distpathrev}}
\newcommand{\pptydistancebi}{\mt{\distpathbi}}
\newcommand{\pptyhascycle}{\mt{\exists\cycle}}
\newcommand{\pptyexactactcycle}{\mt{\{\cycle_{\action,n}\}}}
\newcommand{\pptycycleset}{\mt{\cup{\{\state\}_\cycle}}}
\newcommand{\pptyexactcycle}{\mt{\{\cycle_n\}}}
\newcommand{\pptyscc}{\mt{scc}}
\newcommand{\pptybscc}{\mt{bscc}}
\newcommand{\pptyprop}{\mt{f}}
\newcommand{\pptyident}{id}


\newcommand{\gfctatomicprops}{\mt{\grpfct[\pptyatomicprops]}}
\newcommand{\gfctinitstates}{\mt{\grpfct[\pptyinitstates]^\hasppty}}
\newcommand{\gfcthasoutaction}{\mt{\grpfct[\pptyhasoutact]^\hasppty}}
\newcommand{\gfctminoutaction}{\mt{\grpfct[\pptyminoutact{\numoutact}{\outact}]^\hasppty}}
\newcommand{\gfctmaxoutaction}{\mt{\grpfct[\pptymaxoutact{\numoutact}{\outact}]^\hasppty}}
\newcommand{\gfctspanoutaction}{\mt{\grpfct[\pptyspanoutact{\numoutactb}{\outact}{\numoutact}]^\hasppty}}
\newcommand{\gfctoutactsetsize}{\mt{\grpfct[\pptyoutactsetsize]}}
\newcommand{\gfctoutactsingle}{\mt{\grpfct[\pptyoutactsingle]^\notppty}}
\newcommand{\gfctstrongoutactident}{\mt{\grpfct[\pptystrongoutactident]}}
\newcommand{\gfctweakoutactident}{\mt{\grpfct[\pptyweakoutactident]}}
\newcommand{\gfcthasinaction}{\mt{\grpfct[\pptyhasinact]^\hasppty}}
\newcommand{\gfctmininaction}{\mt{\grpfct[\pptymininact{\numinact}{\inact}]^\hasppty}}
\newcommand{\gfctmaxinaction}{\mt{\grpfct[\pptymaxinact{\numinact}{\inact}]^\hasppty}}
\newcommand{\gfctspaninaction}{\mt{\grpfct[\pptyspaninact{\numinactb}{\inact}{\numinact}]^\hasppty}}
\newcommand{\gfctinactsetsize}{\mt{\grpfct[\pptyinactsetsize]}}
\newcommand{\gfctinactsingle}{\mt{\grpfct[\pptyinactsingle]^\notppty}}
\newcommand{\gfctstronginactident}{\mt{\grpfct[\pptystronginactident]}}
\newcommand{\gfctweakinactident}{\mt{\grpfct[\pptyweakinactident]}}
\newcommand{\gfctparamvalueseq}{\mt{\grpfct[\pptyparamvalueseq]^\hasppty}}
\newcommand{\gfctparamvaluesneq}{\mt{\grpfct[\pptyparamvaluesneq]^\hasppty}}
\newcommand{\gfctparamdnf}{\mt{\grpfct[\pptyparamdnf]^\hasppty}}
\newcommand{\gfctparamcnf}{\mt{\grpfct[\pptyparamcnf]^\hasppty}}
\newcommand{\gfctparamvalueseqopt}{\mt{\pptyparamvalueseqopt}}
\newcommand{\gfctparamvalident}{\mt{\grpfct[\pptyparamvalident]}}
\newcommand{\gfctdistance}{\mt{\grpfct[\pptydistance]}}
\newcommand{\gfctdistancerev}{\mt{\grpfct[\pptydistancerev]}}
\newcommand{\gfctdistancebi}{\mt{\grpfct[\pptydistancebi]}}
\newcommand{\gfcthascycle}{\mt{\grpfct[\pptyhascycle]}}
\newcommand{\gfctexactcycle}{\mt{\grpfct[\pptyexactcycle]}}
\newcommand{\gfctcycleset}{\mt{\grpfct[\pptycycleset]}}
\newcommand{\gfctexactactcycle}{\mt{\grpfct[\pptyexactactcycle]}}
\newcommand{\gfctscc}{\mt{\grpfct[\pptyscc]}}
\newcommand{\gfctbscc}{\mt{\grpfct[\pptybscc]}}
\newcommand{\gfctprop}{\mt{\grpfct[\pptyprop]}}
\newcommand{\gfctident}{\mt{\grpfct[\pptyident]}}

\newcommand{\gfctsubatomicprops}{\mt{\grpfctsub[\pptyatomicprops]}}
\newcommand{\gfctsubinitstates}{\mt{\grpfctsub[\pptyinitstates]^\hasppty}}
\newcommand{\gfctsubhasoutaction}{\mt{\grpfctsub[\pptyhasoutact]^\hasppty}}
\newcommand{\gfctsubminoutaction}{\mt{\grpfctsub[\pptyminoutact{\numoutact}{\outact}]^\hasppty}}
\newcommand{\gfctsubmaxoutaction}{\mt{\grpfctsub[\pptymaxoutact{\numoutact}{\outact}]^\hasppty}}
\newcommand{\gfctsubspanoutaction}{\mt{\grpfctsub[\pptyspanoutact{\numoutactb}{\outact}{\numoutact}]^\hasppty}}
\newcommand{\gfctsuboutactsetsize}{\mt{\grpfctsub[\pptyoutactsetsize]}}
\newcommand{\gfctsuboutactsingle}{\mt{\grpfctsub[\pptyoutactsingle]^\notppty}}
\newcommand{\gfctsubstrongoutactident}{\mt{\grpfctsub[\pptystrongoutactident]^\hasppty}}
\newcommand{\gfctsubweakoutactident}{\mt{\grpfctsub[\pptyweakoutactident]^\hasppty}}
\newcommand{\gfctsubhasinaction}{\mt{\grpfctsub[\pptyhasinact]}}
\newcommand{\gfctsubmininaction}{\mt{\grpfctsub[\pptymininact{\numinact}{\inact}]}}
\newcommand{\gfctsubmaxinaction}{\mt{\grpfctsub[\pptymaxinact{\numinact}{\inact}]}}
\newcommand{\gfctsubspaninaction}{\mt{\grpfctsub[\pptyspaninact{\numinactb}{\inact}{\numinact}]}}
\newcommand{\gfctsubinactsetsize}{\mt{\grpfctsub[\pptyinactsetsize]^\hasppty}}
\newcommand{\gfctsubinactsingle}{\mt{\grpfctsub[\pptyinactsingle]^\notppty}}
\newcommand{\gfctsubstronginactident}{\mt{\grpfctsub[\pptystronginactident]}}
\newcommand{\gfctsubweakinactident}{\mt{\grpfctsub[\pptyweakinactident]}}
\newcommand{\gfctsubparamvalueseq}{\mt{\grpfctsub[\pptyparamvalueseq]^\hasppty}}
\newcommand{\gfctsubparamvaluesneq}{\mt{\grpfctsub[\pptyparamvaluesneq]^\hasppty}}
\newcommand{\gfctsubparamdnf}{\mt{\grpfctsub[\pptyparamdnf]^\hasppty}}
\newcommand{\gfctsubparamcnf}{\mt{\grpfctsub[\pptyparamcnf]^\hasppty}}
\newcommand{\gfctsubparamvalueseqopt}{\mt{\pptyparamvalueseqopt}}
\newcommand{\gfctsubparamvalident}{\mt{\grpfctsub[\pptyparamvalident]}}
\newcommand{\gfctsubdistance}{\mt{\grpfctsub[\pptydistance]}}
\newcommand{\gfctsubdistancerev}{\mt{\grpfctsub[\pptydistancerev]}}
\newcommand{\gfctsubdistancebi}{\mt{\grpfctsub[\pptydistancebi]}}
\newcommand{\gfctsubhascycle}{\mt{\grpfctsub[\pptyhascycle]^\hasppty}}
\newcommand{\gfctsubexactcycle}{\mt{\grpfctsub[\pptyexactcycle]}}
\newcommand{\gfctsubcycleset}{\mt{\grpfctsub[\pptycycleset]}}
\newcommand{\gfctsubexactactcycle}{\mt{\grpfctsub[\pptyexactactcycle]}}
\newcommand{\gfctsubscc}{\mt{\grpfctsub[\pptyscc]}}
\newcommand{\gfctsubbscc}{\mt{\grpfctsub[\pptybscc]}}
\newcommand{\gfctsubprop}{\mt{\grpfctsub[\pptyprop]}}
\newcommand{\gfctsubident}{\mt{\grpfctsub[\pptyident]}}


\newcommand{\viewatomicprops}{\mt{\view[\pptyatomicprops]}}
\newcommand{\viewinitstates}{\mt{\view[\pptyinitstates]^\hasppty}}
\newcommand{\viewhasoutaction}{\mt{\view[\pptyhasoutact]^\hasppty}}
\newcommand{\viewminoutaction}{\mt{\view[\pptyminoutact{\numoutact}{\outact}]^\hasppty}}
\newcommand{\viewmaxoutaction}{\mt{\view[\pptymaxoutact{\numoutact}{\outact}]^\hasppty}}
\newcommand{\viewspanoutaction}{\mt{\view[\pptyspanoutact{\numoutactb}{\outact}{\numoutact}]^\hasppty}}
\newcommand{\viewoutactsetsize}{\mt{\view[\pptyoutactsetsize]}}
\newcommand{\viewoutactsingle}{\mt{\view[\pptyoutactsingle]^\notppty}}
\newcommand{\viewstrongoutactident}{\mt{\view[\pptystrongoutactident]}}
\newcommand{\viewweakoutactident}{\mt{\view[\pptyweakoutactident]}}
\newcommand{\viewhasinaction}{\mt{\view[\pptyhasinact]^\hasppty}}
\newcommand{\viewmininaction}{\mt{\view[\pptymininact{\numinact}{\inact}]^\hasppty}}
\newcommand{\viewmaxinaction}{\mt{\view[\pptymaxinact{\numinact}{\inact}]^\hasppty}}
\newcommand{\viewspaninaction}{\mt{\view[\pptyspaninact{\numinactb}{\inact}{\numinact}]^\hasppty}}
\newcommand{\viewinactsetsize}{\mt{\view[\pptyinactsetsize]}}
\newcommand{\viewinactsingle}{\mt{\view[\pptyinactsingle]^\notppty}}
\newcommand{\viewstronginactident}{\mt{\view[\pptystronginactident]}}
\newcommand{\viewweakinactident}{\mt{\view[\pptyweakinactident]}}
\newcommand{\viewparamvalueseq}{\mt{\view[\pptyparamvalueseq]}}
\newcommand{\viewparamvaluesneq}{\mt{\view[\pptyparamvaluesneq]}}
\newcommand{\viewparamdnf}{\mt{\view[\pptyparamdnf]^\hasppty}}
\newcommand{\viewparamcnf}{\mt{\view[\pptyparamcnf]^\hasppty}}
\newcommand{\viewparamvalueseqopt}{\mt{\pptyparamvalueseqopt}}
\newcommand{\viewparamvalident}{\mt{\view[\pptyparamvalident]}}
\newcommand{\viewdistance}{\mt{\view[\pptydistance]}}
\newcommand{\viewdistancerev}{\mt{\view[\pptydistancerev]}}
\newcommand{\viewdistancebi}{\mt{\view[\pptydistancebi]}}
\newcommand{\viewhascycle}{\mt{\view[\pptyhascycle]}}
\newcommand{\viewexactcycle}{\mt{\view[\pptyexactcycle]}}
\newcommand{\viewcycleset}{\mt{\view[\pptycycleset]}}
\newcommand{\viewexactactcycle}{\mt{\view[\pptyexactactcycle]}}
\newcommand{\viewscc}{\mt{\view[\pptyscc]}}
\newcommand{\viewbscc}{\mt{\view[\pptybscc]}}
\newcommand{\viewprop}{\mt{\view[\pptyprop]}}
\newcommand{\viewident}{\mt{\view[\pptyident]}}

%actions
\newcommand{\numoutact}{\mt{n}}
\newcommand{\numoutactb}{\mt{m}}
\newcommand{\numinact}{\mt{n}}
\newcommand{\numinactb}{\mt{m}}

\newcommand{\predmaxoutact}[1][\numoutact]{\mt{Q_{\outact\leq#1}(\state,\state_1, \dots, \state_{#1+1})}}
\newcommand{\predminoutact}[1][\numoutact]{\mt{Q_{#1\leq\outact}(\state,\state_1, \dots, \state_{#1})}}
\newcommand{\formoutact}[1][\state]{\mt{C_{#1,\outact}}}
\newcommand{\predmaxinact}[1][\numinact]{\mt{Q_{\inact\leq#1}(\state,\state_1, \dots, \state_{#1+1})}}
\newcommand{\predmininact}[1][\numinact]{\mt{Q_{#1\leq\inact}(\state,\state_1, \dots, \state_{#1})}}

\newcommand{\outact}[1][\action]{\mt{\overrightarrow{#1}}}
\newcommand{\outacts}{\mt{\mbox{\smaller[2]$\overrightarrow{\actions}$}}}
\newcommand{\inact}{\mt{\overleftarrow{\action}}}
\newcommand{\inacts}[1][\action]{\mt{\mbox{\smaller[2]$\overleftarrow{#1}$}}}

%%Parameters
\newcommand{\vars}[1][\mdp]{\mt{V\!ar_{#1}}}
\newcommand{\var}{\mt{x}}
\newcommand{\varstate}[1][]{\mt{\var_{\state#1}}}
\newcommand{\varval}{\mt{a}}
\newcommand{\vareval}[1][\mdp]{\mt{V\!arEval_{#1}}}
\newcommand{\varevalimg}[1][\mdp]{\mt{\vareval[#1][\states,\vars]}}
\newcommand{\varevalimgset}{\mt{\arbset}}
\newcommand{\someparam}{\mt{\tilde{x}}}
\newcommand{\eqorneq}{\mt{\;\doteqdot\;}}
\newcommand{\varstyle}[2]{\mt{\langle#1,#2\rangle}}




%\makeatletter
%\newcommand{\overleftrightsmallarrow}{\mathpalette{\overarrowsmall@\leftrightarrowfill@}}
%\newcommand{\overrightsmallarrow}{\mathpalette{\overarrowsmall@\rightarrowfill@}}
%\newcommand{\overleftsmallarrow}{\mathpalette{\overarrowsmall@\leftarrowfill@}}
%\newcommand{\overarrowsmall@}[3]{%
%	\vbox{%
%		\ialign{%
%			##\crcr
%			#1{\smaller@style{#2}}\crcr
%			\noalign{\nointerlineskip}%
%			$\m@th\hfil#2#3\hfil$\crcr
%		}%
%	}%
%}
%\def\smaller@style#1{%
%	\ifx#1\displaystyle\scriptstyle\else
%	\ifx#1\textstyle\scriptstyle\else
%	\scriptscriptstyle
%	\fi
%	\fi
%}
%\makeatother
%\newcommand{\te}[1]{\overleftrightsmallarrow{#1}}

% Distance
\newcommand{\fctdist}{\mt{distance}}
\newcommand{\fctdistdefault}{\mt{\fctdist(\chgph, \smstates, \grandist)}}
\newcommand{\distval}{\mt{d}}
\newcommand{\grandist}{\mt{n}}
\let\path\oldpath
\newcommand{\path}{\mt{P}}
\newcommand{\pathbi}{\mt{\bar{\path}}}
\newcommand{\pathsecfull}{\mt{(\state_1, \action_1, \state_2, \action_2, \dots, \action_{n}, \state_{n+1})}}
\newcommand{\lenpath}{\mt{len}}
\newcommand{\pfirst}{\mt{first}}
\newcommand{\plast}{\mt{last}}
\newcommand{\pathset}{\mt{\path_\chgph}}
\newcommand{\pathbiset}{\mt{\pathbi_\chgph}}
%\newcommand{\distpath}{\mt{\mbox{\smaller[2]$\overrightarrow{dist}$}}}
\newcommand{\distpath}{\mt{dist}}
\newcommand{\distpathrev}{\mt{\mbox{\smaller[2]$\overleftarrow{dist}$}}}
%\newcommand{\distpathbi}{\mt{\mbox{\smaller[2]$\overline{dist}$}}}
\newcommand{\distpathbi}{\mt{\overline{dist}}}
%Cycles
\newcommand{\cyclesecfull}{\mt{(\state_0, \action_0, \state_1, \action_1, \dots, \action_{n-1}, \state_0)}}
\newcommand{\fctfindcycles}{\mt{findCycles}}
\newcommand{\cycle}{\mt{C}}
\newcommand{\cycleset}{\mt{\cycle_{\mdp, n}}}
\newcommand{\lencycle}{\mt{len}}
% strongly connected components
\newcommand{\scc}{\mt{T}}
\newcommand{\setscc}{\mt{SCC_{\chgph,n}}}
\newcommand{\setbscc}{\mt{BSCC_{\chgph,n}}}

% properties
\newcommand{\propfct}{\mt{f}}

% all Systems
\newcommand{\chgph}{\mt{\mdp}}
\newcommand{\chgphtuple}{\mt{\mdptuple}}
\newcommand{\chgphtupledist}{\mt{\mdptupledist}}

\newcommand{\states}{\mt{S}}
\newcommand{\actions}{\mt{Act}}
\newcommand{\atomicprops}{\mt{AP}}
\newcommand{\labelingfct}{\mt{L}}
\newcommand{\init}{\mt{\initdistrib}} % use MDP % refers to the underlying set
\newcommand{\trans}{\mt{\probtfunc}} % use MDP % refers to the underlying set
\newcommand{\smstates}{\mt{\tilde{\states}}}


\newcommand{\state}{\mt{s}}
\newcommand{\action}{\mt{\alpha}}
\newcommand{\actionb}{\mt{\beta}}
\newcommand{\actionc}{\mt{\gamma}}
\newcommand{\smstate}{\mt{\tilde{\state}}}



% transition sysstems
\newcommand{\ts}{\mt{TS}}
\newcommand{\transitionrel}{\mt{\longrightarrow}}
\newcommand{\initstates}{\mt{I}}
\newcommand{\transitionsystem}{\mt
	{(\states, \actions, \transitionrel, \initstates, \atomicprops, \labelingfct)}
}
\newcommand{\tstupledist}{\mt{(\states', \actions',\transitionrel', \initstates', \labelingfct')}}


%Markov chains and MDP
\newcommand{\mdp}{\mt{\autm}}
\newcommand{\mdptuple}{\mt{(\states, \actions, \probtfunc, \initdistrib, \atomicprops, \labelingfct)}}
\newcommand{\mdptupledist}{\mt{(\states', \actions', \probtfunc', \initdistrib', \atomicprops', \labelingfct')}}
\newcommand{\autm}{\mt{\mathcal{M}}}
\newcommand{\probtfunc}{\mt{\textbf{P}}}
\newcommand{\initdistrib}{\mt{\iota_{init}}}


%maths
\newcommand{\powerset}[1]{\mt{\mathcal{P}(#1)}}
\newcommand{\eqclass}[2]{\mt{[#1]_{#2}}}%{\mt{#1 / #2}}
\newcommand{\impr}{\mt{\hspace{3mm}\Rightarrow\hspace{2mm}}}
\newcommand{\impl}{\mt{\hspace{3mm}\Leftarrow\hspace{2mm}}}
\newcommand{\natnums}{\mt{\mathbb{N}}} 
\newcommand{\realnums}{\mt{\mathbb{R}}}
\newcommand{\integers}{\mt{\mathbb{Z}}}
\newcommand{\intmodn}[1][n]{\mt{\mathbb{Z}_{#1}}}
\newcommand{\arbset}{\mt{M}}
\newcommand{\bigsum}[2][]{\mt{\mathlarger{\sum}_{#2}^{#1}}}
\newcommand{\bbigsum}[2][]{\mt{\mathlarger{\mathlarger{\sum}}_{#2}^{#1}}}
\newcommand{\invimage}[2]{#1^{\mt{-1}(#2)}}
\newcommand{\img}{\mt{Img}}
\newcommand{\cond}{\mt{\,|\,}}

%tickz
%% \definecolor{darkred}{RGB}{196, 42, 42}

%implementation
\newcommand{\pmcvis}{\nm{PMC-Vis}}
\newcommand{\mytexttilde}{\raisebox{0.5ex}{\texttildelow}}%{\mbox{\smaller[2]$\sim$}}
