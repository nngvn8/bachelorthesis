
%%%%%%%%%%%%%%%%%%%%%%%%%%%%%%%%%%%% PACKAGES %%%%%%%%%%%%%%%%%%%%%%%%%%%%%%%%%%%%%%%%%%

\usepackage{inputenc,fontenc}
\usepackage[a4paper,margin=3cm]{geometry}
\usepackage[english,german]{babel}
%\usepackage[german]{babel}
%\usepackage[fixlanguage]{babelbib}


\usepackage{bbold}
\usepackage{amsthm}
\usepackage{amsmath}
\usepackage[dvipsnames]{xcolor}
\usepackage{standalone}
\usepackage{tikz}
\usepackage{cite}
\usepackage{xspace}
\usepackage{relsize}


%%%%%%%%%%%%%%%%%%%%%%%%%%%%%%%%%%%% THEOREMSTYLES %%%%%%%%%%%%%%%%%%%%%%%%%%%%%%%%%%

\theoremstyle{definition}
\newtheorem{definition}{Definition}[section]

\theoremstyle{theorem}
\newtheorem{theorem}{Satz}[section]

\theoremstyle{korollary}
\newtheorem{korollary}{Korollar}[section]

\theoremstyle{definition}
\newtheorem{exmp}{Beispiel}[section]


%%%%%%%%%%%%%%%%%%%%%%%%%%%%%%%%%%% MY MACROS %%%%%%%%%%%%%%%%%%%%%%%%%%%%%%%%%%%%%%%%%
%formatting
\newcommand{\comment}[2]{{\color{#1}#2}}
\newcommand{\redcomment}[1]{{\color{red}#1}}
\newcommand{\mt}[1]{\ensuremath{{#1}}\xspace}
\newcommand{\mynewcommand}[2]{\newcommand{#1}{\mt{#2}}} %% currently not used becaue of ide highlighting

%namings
\newcommand{\nm}[1]{#1\xspace}
\newcommand{\viewN}{\nm{view}}
\newcommand{\viewNC}{\nm{View}}
\newcommand{\grpfctN}{\nm{grouping function}}
\newcommand{\grpfctNC}{\nm{Grouping function}}
\newcommand{\grpfctNCC}{\nm{Grouping Function}}

%shortcuts
\newcommand{\ts}{\nm{TS}}

%Identifiers for self created definitions
\newcommand{\grpfct}{\mt{F}}
\newcommand{\view}[2]{\mt{#1_#2}}
\newcommand{\imggrp}{\mt{\arbset}}
\newcommand{\viewppty}{\mt{\grpfct}}
\newcommand{\viewid}{\mt{\ts}}




%Transitionsystem
\newcommand{\state}{\mt{s}}
\newcommand{\action}{\mt{\alpha}}
\newcommand{\actionb}{\mt{\beta}}
\newcommand{\actionc}{\mt{\alpha}}
\newcommand{\atomicprop}{\mt{ap}}

\newcommand{\smstate}{\mt{\tilde{\state}}}

\newcommand{\states}{\mt{S}}
\newcommand{\actions}{\mt{Act}}
\newcommand{\transitionrel}{\mt{\longrightarrow}}
\newcommand{\initstates}{\mt{I}}
\newcommand{\atomicprops}{\mt{AP}}
\newcommand{\labelingfct}{\mt{L}}
\newcommand{\transitionsystem}{\mt
	{(\states, \actions, \transitionrel, \initstates, \atomicprops, \labelingfct)}
}
\newcommand{\eqrelview}{\mt{R}}
\newcommand{\eqclassv}[1]{\mt{\eqclass{#1}{\eqrelview}}}

%Markov chains
\newcommand{\autm}{\mt{\mathcal{M}}}
\newcommand{\probtfunc}{\mt{\textbf{P}}}
\newcommand{\initdist}{\redcomment{\mt{l_{init}}}}



%maths
\newcommand{\powerset}[1]{\mt{\mathcal{P}(#1)}}
\newcommand{\eqclass}[2]{\mt{[#1]_{#2}}}%{\mt{#1 / #2}}
\newcommand{\impr}{\mt{\hspace{3mm}\Rightarrow\hspace{2mm}}}
\newcommand{\impl}{\mt{\hspace{3mm}\Leftarrow\hspace{2mm}}}
\newcommand{\natnums}{\mt{\mathbb{N}}} 
\newcommand{\arbset}{\mt{M}}
\newcommand{\bigsum}[2][]{\mt{\mathlarger{\sum}_{#2}^{#1}}}
\newcommand{\bbigsum}[2][]{\mt{\mathlarger{\mathlarger{\sum}}_{#2}^{#1}}}

%tickz
%% \definecolor{darkred}{RGB}{196, 42, 42}
