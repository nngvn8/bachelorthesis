
%%%%%%%%%%%%%%%%%%%%%%%%%%%%%%%%%%%% PACKAGES %%%%%%%%%%%%%%%%%%%%%%%%%%%%%%%%%%%%%%%%%%

\usepackage{inputenc,fontenc}
\usepackage[a4paper,margin=3cm]{geometry}
\usepackage[english,german]{babel}
%\usepackage[german]{babel}
%\usepackage[fixlanguage]{babelbib}


\usepackage{bbold}
\usepackage{amsthm}
\usepackage{amsmath}
\usepackage{amssymb} % doteqdot
\usepackage[dvipsnames]{xcolor}
\usepackage{standalone}
\usepackage{tikz}
\usepackage{cite}
\usepackage{xspace}
\usepackage{relsize}
\usepackage{mathtools} %mathclap
\usepackage{hyperref}
\hypersetup{
	colorlinks,
	citecolor=black,
	filecolor=black,
	linkcolor=black,
	urlcolor=black
}
%\usepackage{xparse} % \newDocumentCommand for multiple optional arguments
%\usepackage{titlecaps}


%%%%%%%%%%%%%%%%%%%%%%%%%%%%%%%%%%%% THEOREMSTYLES %%%%%%%%%%%%%%%%%%%%%%%%%%%%%%%%%%

\theoremstyle{definition}
\newtheorem{definition}{Definition}[section]

\theoremstyle{theorem}
\newtheorem{theorem}{Satz}[section]

\theoremstyle{korollary}
\newtheorem{korollary}{Korollar}[section]

\theoremstyle{definition}
\newtheorem{exmp}{Beispiel}[section]


%%%%%%%%%%%%%%%%%%%%%%%%%%%%%%%%%%% MY MACROS %%%%%%%%%%%%%%%%%%%%%%%%%%%%%%%%%%%%%%%%%
%formatting
\newcommand{\comment}[2]{{\color{#1}#2}}
\newcommand{\redcomment}[1]{{\color{red}#1}}
\newcommand{\purpcomment}[1]{{\color{pink}#1}}
\newcommand{\bluecomment}[1]{{\color{blue}#1}}
\newcommand{\mt}[1]{\ensuremath{{#1}}\xspace}
\newcommand{\mynewcommand}[2]{\newcommand{#1}{\mt{#2}}} %% currently not used becaue of ide highlighting

%own names
\newcommand{\nm}[1]{#1\xspace}
\newcommand{\mdpN}{\nm{MDP}}
\newcommand{\mdpsN}{\nm{MDPs}}
\newcommand{\viewN}{\nm{view}}
\newcommand{\viewNC}{\nm{View}}
\newcommand{\viewsN}{\nm{views}}
\newcommand{\viewsNC}{\nm{Views}}
\newcommand{\grpfctN}{\nm{grouping function}}
\newcommand{\grpfctNC}{\nm{Grouping function}}
\newcommand{\grpfctNCC}{\nm{Grouping Function}}
\newcommand{\grpfctsN}{\nm{grouping functions}}
\newcommand{\grpfctsNC}{\nm{Grouping functions}}
\newcommand{\grpfctsNCC}{\nm{Grouping Functions}}
%\newcommand{\chosengraphtypeNCC}{\nm{Transition System}}
%\newcommand{\chosengraphtypeNC}{\nm{Transition system}}
%\newcommand{\chosengraphtypeN}{\nm{transition system}}
%\newcommand{\chosengraphtypesNCC}{\nm{Transition Systems}}
%\newcommand{\chosengraphtypesNC}{\nm{Transition systems}}
%\newcommand{\chosengraphtypesN}{\nm{transition systems}}
\newcommand{\chosengraphtypeNCC}{\nm{MDP}}
\newcommand{\chosengraphtypeNC}{\nm{MDP}}
\newcommand{\chosengraphtypeN}{\nm{an MDP}}
\newcommand{\chosengraphtypesNCC}{\nm{MDPs}}
\newcommand{\chosengraphtypesNC}{\nm{MDPs}}
\newcommand{\chosengraphtypesN}{\nm{MDPs}}
\newcommand{\parllcompN}{\nm{parallel composition}}
\newcommand{\parllcompNC}{\nm{Parallel composition}}
\newcommand{\parllcompNCC}{\nm{Parallel Composition}}
\newcommand{\parllcompsN}{\nm{parallel compositions}}
\newcommand{\parllcompsNC}{\nm{Parallel compositions}}
\newcommand{\parllcompsNCC}{\nm{Parallel Compositions}}

\newcommand{\outactident}{\nm{OutActionsIdent}}

%names
\newcommand{\iffN}{\nm{if and only if}}
\newcommand{\tsN}{\nm{TS}}

%% outactions identical
\newcommand{\outactidentstrong}{\nm{strong}}
\newcommand{\outactidentweak}{\nm{weak}}

% CORE DEFINITIONS
\newcommand{\grpfct}[1][\viewppty]{\mt{F_{#1}}}
\newcommand{\eqrelview}{\mt{R}}
\newcommand{\eqclassv}[1]{\mt{\eqclass{#1}{\eqrelview}}}
\newcommand{\viewid}{\mt{\mdp}}
\newcommand{\view}[1][\viewppty]{\mt{\viewid_{#1}}}
\newcommand{\imggrp}{\mt{\arbset}}
\newcommand{\viewppty}{\mt{\theta}}
\newcommand{\pll}{\mt{||}}
\newcommand{\remstates}{\mt{\bigcup_{\state \in \states \setminus \states_1}\{\state\}}}

%\newcommand{\mdp}{def}\mdp
%\newcommand{\mdpdef}


% EXAMPLE VIEWS
\newcommand{\gfctatomicprops}{\mt{\grpfct[\atomicprops]}}
\newcommand{\gfctinitstates}{\mt{\grpfct[\initstates]}}
\newcommand{\gfcthasoutaction}{\mt{\grpfct[\exists\outact]}}
\newcommand{\gfctminoutaction}{\mt{\grpfct[\numoutact\leq\outact]}}
\newcommand{\gfctmaxoutaction}{\mt{\grpfct[\outact\leq\numoutact]}}
\newcommand{\gfctspanoutaction}{\mt{\grpfct[\numoutactb\leq\outact\leq\numoutact]}}
\newcommand{\gfctstrongoutactident}{\mt{\grpfct[\outacts(\state)_=]}}
\newcommand{\gfctweakoutactident}{\mt{\grpfct[\outacts(\state)_\approx]}}
\newcommand{\gfcthasinaction}{\mt{\grpfct[\exists\inact]}}
\newcommand{\gfctmininaction}{\mt{\grpfct[\numinact\leq\inact]}}
\newcommand{\gfctmaxinaction}{\mt{\grpfct[\inact\leq\numinact]}}
\newcommand{\gfctspaninaction}{\mt{\grpfct[\numinactb\leq\inact\leq\numinact]}}
\newcommand{\gfctstronginactident}{\mt{\grpfct[\inacts(\state)_=]}}
\newcommand{\gfctweakinactident}{\mt{\grpfct[\inacts(\state)_\approx]}}
\newcommand{\gfctparamvalueseq}{\mt{\grpfct[\param = \paramval]}}
\newcommand{\gfctparamvaluesneq}{\mt{\grpfct[\param \neq \paramval]}}
\newcommand{\gfctparamdnf}{\mt{\grpfct[ParDNF]}}
\newcommand{\gfctparamcnf}{\mt{\grpfct[ParCNF]}}
\newcommand{\gfctparamvalueseqopt}[1][\paramval]{\mt{\grpfct[\param = #1]}}
\newcommand{\gfctparamvalident}{\mt{\grpfct[\parameval(\state,\param)]}}


\newcommand{\viewatomicprops}{\mt{\view[\atomicprops]}}
\newcommand{\viewinitstates}{\mt{\view[\initstates]}}
\newcommand{\viewhasoutaction}{\mt{\view[\exists\outact]}}
\newcommand{\viewminoutaction}{\mt{\view[\numoutact\leq\outact]}}
\newcommand{\viewmaxoutaction}{\mt{\view[\outact\leq\numoutact]}}
\newcommand{\viewspanoutaction}{\mt{\view[\numoutactb\leq\outact\leq\numoutact]}}
\newcommand{\viewstrongoutactident}{\mt{\view[\outacts(\state)_=]}}
\newcommand{\viewweakoutactident}{\mt{\view[\outacts(\state)_\approx]}}
\newcommand{\viewhasinaction}{\mt{\view[\exists\inact]}}
\newcommand{\viewmininaction}{\mt{\view[\numinact\leq\inact]}}
\newcommand{\viewmaxinaction}{\mt{\view[\inact\leq\numinact]}}
\newcommand{\viewspaninaction}{\mt{\view[\numinactb\leq\inact\leq\numinact]}}
\newcommand{\viewstronginactident}{\mt{\view[\inacts(\state)_=]}}
\newcommand{\viewweakinactident}{\mt{\view[\inacts(\state)_\approx]}}
\newcommand{\viewparamvalueseq}{\mt{\view[\param = \paramval]}}
\newcommand{\viewparamvaluesneq}{\mt{\view[\param \neq \paramval]}}
\newcommand{\viewparamdnf}{\mt{\view[ParDNF]}}
\newcommand{\viewparamcnf}{\mt{\view[ParCNF]}}
\newcommand{\viewparamvalident}{\mt{\view[\parameval(\state,\param)]}}


%\newcommand{\viewatomicprops}{\mt{\view{\gfctatomicprops}}}
%\newcommand{\viewinitstates}{\mt{\view{\gfctinitstates}}}
%\newcommand{\viewhasoutaction}{\mt{\view{\gfcthasoutaction}}}
%\newcommand{\viewminoutaction}{\mt{\view{\gfctminoutaction}}}
%\newcommand{\viewmaxoutaction}{\mt{\view{\gfctmaxoutaction}}}
%\newcommand{\viewspanoutaction}{\mt{\view{\gfctspanoutaction}}}
%\newcommand{\viewstrongoutactident}{\mt{\view{\gfctstrongoutactident}}}
%\newcommand{\viewweakoutactident}{\mt{\view{\gfctweakoutactident}}}
%\newcommand{\viewhasinaction}{\mt{\view{\gfcthasinaction}}}
%\newcommand{\viewmininaction}{\mt{\view{\gfctmininaction}}}
%\newcommand{\viewmaxinaction}{\mt{\view{\gfctmaxinaction}}}
%\newcommand{\viewspaninaction}{\mt{\view{\gfctspaninaction}}}
%\newcommand{\viewstronginactident}{\mt{\view{\gfctstronginactident}}}
%\newcommand{\viewweakinactident}{\mt{\view{\gfctweakinactident}}}
%\newcommand{\viewparamvalueseq}{\mt{\view{\gfctparamvalueseq}}}
%\newcommand{\viewparamvaluesneq}{\mt{\view{\gfctparamvaluesneq}}}
%\newcommand{\viewparamdnf}{\mt{\view{\gfctparamdnf}}}
%\newcommand{\viewparamcnf}{\mt{\view{\gfctparamcnf}}}
%\newcommand{\viewparamvalident}{\mt{\view{\gfctparamvalident}}}

%actions
\newcommand{\numoutact}{\mt{n}}
\newcommand{\numoutactb}{\mt{m}}
\newcommand{\numinact}{\mt{n}}
\newcommand{\numinactb}{\mt{m}}
\newcommand{\setoutact}{\mt{\actions \cup \states}}

\newcommand{\predmaxoutact}[1][\numoutact]{\mt{Q_{\outact\leq#1}(\state,\state_1, \dots, \state_{#1+1})}}
\newcommand{\predminoutact}[1][\numoutact]{\mt{Q_{#1\leq\outact}(\state,\state_1, \dots, \state_{#1})}}
\newcommand{\formoutact}[1][\state]{\mt{C_{#1,\outact}}}
\newcommand{\predmaxinact}[1][\numinact]{\mt{Q_{\inact\leq#1}(\state,\state_1, \dots, \state_{#1+1})}}
\newcommand{\predmininact}[1][\numinact]{\mt{Q_{#1\leq\inact}(\state,\state_1, \dots, \state_{#1})}}

\newcommand{\outact}{\mt{\overrightarrow{\action}}}
\newcommand{\outacts}{\mt{\overrightarrow{\actions}}}
\newcommand{\inact}{\mt{\overleftarrow{\action}}}
\newcommand{\inacts}{\mt{\overleftarrow{\actions}}}

%%Parameters
\newcommand{\params}{\mt{Par}}
\newcommand{\param}{\mt{x}}
\newcommand{\paramstate}[1][]{\mt{\param_{\state#1}}}
\newcommand{\paramval}{\mt{a}}
\newcommand{\parameval}{\mt{ParEval}}
\newcommand{\paramevalimg}{\mt{\arbset}}
\newcommand{\someparam}{\mt{\tilde{x}}}
\newcommand{\eqorneq}{\mt{\;\doteqdot\;}}

%Cycles
\newcommand{\cyclesecfull}{\mt{(\state_0, \action_0, \state_1, \action_1, \dots, \action_{n-1}, \state_0)}}
\newcommand{\fctfindcycle}{\mt{findCycle}}
\newcommand{\cycle}{\mt{C}}
\newcommand{\cycleset}{\mt{\cycle_{\{\}}}}


% all Systems
\newcommand{\chgph}{\mt{\mdp}}
\newcommand{\chgphtuple}{\mt{\mdptuple}}
\newcommand{\chgphtupledist}{\mt{\mdptupledist}}

\newcommand{\states}{\mt{S}}
\newcommand{\actions}{\mt{Act}}
\newcommand{\atomicprops}{\mt{AP}}
\newcommand{\labelingfct}{\mt{L}}
\newcommand{\init}{\mt{\initdistrib}} % use MDP % refers to the underlying set
\newcommand{\trans}{\mt{\probtfunc}} % use MDP % refers to the underlying set


\newcommand{\state}{\mt{s}}
\newcommand{\action}{\mt{\alpha}}
\newcommand{\actionb}{\mt{\beta}}
\newcommand{\actionc}{\mt{\alpha}}
\newcommand{\smstate}{\mt{\tilde{\state}}}



% transition sysstems
\newcommand{\ts}{\mt{TS}}
\newcommand{\transitionrel}{\mt{\longrightarrow}}
\newcommand{\initstates}{\mt{I}}
\newcommand{\transitionsystem}{\mt
	{(\states, \actions, \transitionrel, \initstates, \atomicprops, \labelingfct)}
}
\newcommand{\tstupledist}{\mt{(\states', \actions',\transitionrel', \initstates', \labelingfct')}}


%Markov chains and MDP
\newcommand{\mdp}{\mt{\autm}}
\newcommand{\mdptuple}{\mt{(\states, \actions, \probtfunc, \initdistrib, \atomicprops, \labelingfct)}}
\newcommand{\mdptupledist}{\mt{(\states', \actions', \probtfunc', \initdistrib', \atomicprops', \labelingfct')}}
\newcommand{\autm}{\mt{\mathcal{M}}}
\newcommand{\probtfunc}{\mt{\textbf{P}}}
\newcommand{\initdistrib}{\mt{\iota_{init}}}


%maths
\newcommand{\powerset}[1]{\mt{\mathcal{P}(#1)}}
\newcommand{\eqclass}[2]{\mt{[#1]_{#2}}}%{\mt{#1 / #2}}
\newcommand{\impr}{\mt{\hspace{3mm}\Rightarrow\hspace{2mm}}}
\newcommand{\impl}{\mt{\hspace{3mm}\Leftarrow\hspace{2mm}}}
\newcommand{\natnums}{\mt{\mathbb{N}}} 
\newcommand{\arbset}{\mt{M}}
\newcommand{\bigsum}[2][]{\mt{\mathlarger{\sum}_{#2}^{#1}}}
\newcommand{\bbigsum}[2][]{\mt{\mathlarger{\mathlarger{\sum}}_{#2}^{#1}}}
\newcommand{\invimage}[2]{#1^{\mt{-1}(#2)}}

%tickz
%% \definecolor{darkred}{RGB}{196, 42, 42}
