\documentclass[preview]{standalone}

%%%%%%%%%%%%%%%%%%%%%%%%%%%%%%%%%%%% PACKAGES %%%%%%%%%%%%%%%%%%%%%%%%%%%%%%%%%%%%%%%%%%

\usepackage{inputenc,fontenc}
\usepackage[a4paper,margin=3cm]{geometry}
\usepackage[english,german]{babel}
%\usepackage[german]{babel}
%\usepackage[fixlanguage]{babelbib}


\usepackage{bbold}
\usepackage{amsthm}
\usepackage{amsmath}
\usepackage[dvipsnames]{xcolor}
\usepackage{standalone}
\usepackage{tikz}
\usepackage{cite}
\usepackage{xspace}
\usepackage{relsize}
\usepackage{mathtools} %mathclap
%\usepackage{xparse} % \newDocumentCommand for multiple optional arguments
%\usepackage{titlecaps}


%%%%%%%%%%%%%%%%%%%%%%%%%%%%%%%%%%%% THEOREMSTYLES %%%%%%%%%%%%%%%%%%%%%%%%%%%%%%%%%%

\theoremstyle{definition}
\newtheorem{definition}{Definition}[section]

\theoremstyle{theorem}
\newtheorem{theorem}{Satz}[section]

\theoremstyle{korollary}
\newtheorem{korollary}{Korollar}[section]

\theoremstyle{definition}
\newtheorem{exmp}{Beispiel}[section]


%%%%%%%%%%%%%%%%%%%%%%%%%%%%%%%%%%% MY MACROS %%%%%%%%%%%%%%%%%%%%%%%%%%%%%%%%%%%%%%%%%
%formatting
\newcommand{\comment}[2]{{\color{#1}#2}}
\newcommand{\redcomment}[1]{{\color{red}#1}}
\newcommand{\purpcomment}[1]{{\color{pink}#1}}
\newcommand{\bluecomment}[1]{{\color{blue}#1}}
\newcommand{\mt}[1]{\ensuremath{{#1}}\xspace}
\newcommand{\mynewcommand}[2]{\newcommand{#1}{\mt{#2}}} %% currently not used becaue of ide highlighting

%own names
\newcommand{\nm}[1]{#1\xspace}
\newcommand{\viewN}{\nm{view}}
\newcommand{\viewNC}{\nm{View}}
\newcommand{\viewsN}{\nm{views}}
\newcommand{\viewsNC}{\nm{Views}}
\newcommand{\grpfctN}{\nm{grouping function}}
\newcommand{\grpfctNC}{\nm{Grouping function}}
\newcommand{\grpfctNCC}{\nm{Grouping Function}}
\newcommand{\grpfctsN}{\nm{grouping functions}}
\newcommand{\grpfctsNC}{\nm{Grouping functions}}
\newcommand{\grpfctsNCC}{\nm{Grouping Functions}}
\newcommand{\chosengraphtypeNCC}{\nm{Transition System}}
\newcommand{\chosengraphtypeNC}{\nm{Transition system}}
\newcommand{\chosengraphtypeN}{\nm{transition system}}
\newcommand{\chosengraphtypesNCC}{\nm{Transition Systems}}
\newcommand{\chosengraphtypesNC}{\nm{Transition systems}}
\newcommand{\chosengraphtypesN}{\nm{transition systems}}
\newcommand{\parllcompN}{\nm{parallel composition}}
\newcommand{\parllcompNC}{\nm{Parallel composition}}
\newcommand{\parllcompNCC}{\nm{Parallel Composition}}

\newcommand{\outactident}{\nm{OutActionsIdent}}

%names
\newcommand{\iffN}{\nm{if and only if}}
\newcommand{\tsN}{\nm{TS}}

%% outactions identical
\newcommand{\outactidentstrong}{\nm{strong}}
\newcommand{\outactidentweak}{\nm{weak}}

%Identifiers for self created definitions
\newcommand{\grpfct}{\mt{F}}
\newcommand{\view}[2][\viewid]{\mt{#1_#2}}
\newcommand{\imggrp}{\mt{\arbset}}
\newcommand{\viewppty}{\mt{\grpfct}}
\newcommand{\viewid}{\mt{\ts}}
\newcommand{\pll}{\mt{||}}
\newcommand{\remstates}{\mt{\bigcup_{\state \in \states \setminus \states_1}\{\state\}}}


%example views
\newcommand{\outact}{\mt{\overrightarrow{\action}}}
\newcommand{\outacts}{\mt{\overrightarrow{\actions}}}
\newcommand{\inact}{\mt{\overleftarrow{\action}}}
\newcommand{\inacts}{\mt{\overleftarrow{\actions(\state)}}}
\newcommand{\gfctatomicprops}{\mt{{\viewppty_\atomicprops}}}
\newcommand{\gfctinitstates}{\mt{{\viewppty_\initstates}}}
\newcommand{\gfcthasoutaction}{\mt{{\viewppty_{\exists\outact}}}}
\newcommand{\gfctminoutaction}{\mt{{\viewppty_{\numoutact\leq\outact}}}}
\newcommand{\gfctmaxoutaction}{\mt{{\viewppty_{\outact\leq\numoutact}}}}
\newcommand{\gfctspanoutaction}{\mt{{\viewppty_{\numoutactb\leq\outact\leq\numoutact}}}}
\newcommand{\gfctstrongoutactident}{\mt{{\viewppty_{\outacts(\state)_=}}}}
\newcommand{\gfctweakoutactident}{\mt{{\viewppty_{\outacts(\state)_\approx}}}}
\newcommand{\gfcthasinaction}{\mt{{\viewppty_{\exists\inact}}}}
\newcommand{\gfctmininaction}{\mt{{\viewppty_{\numinact\leq\inact}}}}
\newcommand{\gfctmaxinaction}{\mt{{\viewppty_{\inact\leq\numinact}}}}
\newcommand{\gfctspaninaction}{\mt{{\viewppty_{\numinactb\leq\inact\leq\numinact}}}}
\newcommand{\gfctstronginactident}{\mt{{\viewppty_{\inacts(\state)_=}}}}
\newcommand{\gfctweakinactident}{\mt{{\viewppty_{\intacts(\state)_\approx}}}}
\newcommand{\gfctparamvalueseq}{\mt{\viewppty_{\param = \paramval}}}
\newcommand{\gfctparamvaluesneq}{\mt{\viewppty_{\param \neq \paramval}}}
\newcommand{\gfctparamvalueseqopt}[1][\paramval]{\mt{\viewppty_{\param = #1}}}
\newcommand{\gfctparamvalident}{\mt{\viewppty_{\parameval(\state,\param)}}}


\newcommand{\viewatomicprops}{\mt{\view{\gfctatomicprops}}}
\newcommand{\viewinitstates}{\mt{\view{\gfctinitstates}}}
\newcommand{\viewhasoutaction}{\mt{\view{\gfcthasoutaction}}}
\newcommand{\viewminoutaction}{\mt{\view{\gfctminoutaction}}}
\newcommand{\viewmaxoutaction}{\mt{\view{\gfctmaxoutaction}}}
\newcommand{\viewspanoutaction}{\mt{\view{\gfctspanoutaction}}}
\newcommand{\viewstrongoutactident}{\mt{\view{\gfctstrongoutactident}}}
\newcommand{\viewweakoutactident}{\mt{\view{\gfctweakoutactident}}}
\newcommand{\viewhasinaction}{\mt{\view{\gfcthasinaction}}}
\newcommand{\viewmininaction}{\mt{\view{\gfctmininaction}}}
\newcommand{\viewmaxinaction}{\mt{\view{\gfctmaxinaction}}}
\newcommand{\viewspaninaction}{\mt{\view{\gfctspaninaction}}}
\newcommand{\viewstronginactident}{\mt{\view{\gfctstronginactident}}}
\newcommand{\viewweakinactident}{\mt{\view{\gfctweakinactident}}}
\newcommand{\viewparamvalueseq}{\mt{\view{\gfctparamvalueseq}}}
\newcommand{\viewparamvaluesneq}{\mt{\view{\gfctparamvaluesneq}}}
\newcommand{\viewparamvalident}{\mt{\view{\gfctparamvalident}}}

%%OutAct
\newcommand{\numoutact}{\mt{n}}
\newcommand{\numoutactb}{\mt{m}}
\newcommand{\numinact}{\mt{n}}
\newcommand{\numinactb}{\mt{m}}
\newcommand{\setoutact}{\mt{\actions \cup \states}}

\newcommand{\predmaxoutact}[1][\numoutact]{\mt{Q_{\outact\leq#1}(\state,\state_1, \dots, \state_{#1+1})}}
\newcommand{\predminoutact}[1][\numoutact]{\mt{Q_{#1\leq\outact}(\state,\state_1, \dots, \state_{#1})}}
\newcommand{\formoutact}[1][\state]{\mt{C_{#1,\outact}}}
\newcommand{\predmaxinact}[1][\numinact]{\mt{Q_{\inact\leq#1}(\state,\state_1, \dots, \state_{#1+1})}}
\newcommand{\predmininact}[1][\numinact]{\mt{Q_{#1\leq\inact}(\state,\state_1, \dots, \state_{#1})}}

%%Parameters
\newcommand{\params}{\mt{Par}}
\newcommand{\param}{\mt{x}}
\newcommand{\paramval}{\mt{a}}
\newcommand{\parameval}{\mt{ParEval}}
\newcommand{\paramevalimg}{\mt{\arbset}}
\newcommand{\someparam}{\mt{\tilde{x}}}

%Transitionsystem
\newcommand{\ts}{\mt{TS}}
\newcommand{\state}{\mt{s}}
\newcommand{\action}{\mt{\alpha}}
\newcommand{\actionb}{\mt{\beta}}
\newcommand{\actionc}{\mt{\alpha}}
\newcommand{\atomicprop}{\mt{ap}}

\newcommand{\smstate}{\mt{\tilde{\state}}}

\newcommand{\states}{\mt{S}}
\newcommand{\actions}{\mt{Act}}
\newcommand{\transitionrel}{\mt{\longrightarrow}}
\newcommand{\initstates}{\mt{I}}
\newcommand{\atomicprops}{\mt{AP}}
\newcommand{\labelingfct}{\mt{L}}
\newcommand{\transitionsystem}{\mt
	{(\states, \actions, \transitionrel, \initstates, \atomicprops, \labelingfct)}
}
\newcommand{\eqrelview}{\mt{R}}
\newcommand{\eqclassv}[1]{\mt{\eqclass{#1}{\eqrelview}}}

%Markov chains and MDP
\newcommand{\autm}{\mt{\mathcal{M}}}
\newcommand{\probtfunc}{\mt{\textbf{P}}}
\newcommand{\initdist}{\redcomment{\mt{l_{init}}}}



%maths
\newcommand{\powerset}[1]{\mt{\mathcal{P}(#1)}}
\newcommand{\eqclass}[2]{\mt{[#1]_{#2}}}%{\mt{#1 / #2}}
\newcommand{\impr}{\mt{\hspace{3mm}\Rightarrow\hspace{2mm}}}
\newcommand{\impl}{\mt{\hspace{3mm}\Leftarrow\hspace{2mm}}}
\newcommand{\natnums}{\mt{\mathbb{N}}} 
\newcommand{\arbset}{\mt{M}}
\newcommand{\bigsum}[2][]{\mt{\mathlarger{\sum}_{#2}^{#1}}}
\newcommand{\bbigsum}[2][]{\mt{\mathlarger{\mathlarger{\sum}}_{#2}^{#1}}}
\newcommand{\invimage}[2]{#1^{\mt{-1}(#2)}}

%tickz
%% \definecolor{darkred}{RGB}{196, 42, 42}


\begin{document}
	
	\section{Preliminaries}
	
	\subsection{Transition Systems}
	\redcomment{Motivation of transition systems} 
	
	\redcomment{The following definition is directly taken form Principles of Modelchecking, Baier p. 20}
	
	\begin{definition}
		A \emph{transition system} TS is a tuple \transitionsystem where
		\begin{itemize}
			\item \states is a set of states,
			\item \actions is a set of actions,
			\item $\transitionrel \subseteq \states \times \actions \times \states$ is transition relation,
			\item $\initstates \subseteq \states$ is a set of initial states,
			\item \atomicprops is a set of atomic propositions, and
			\item $\labelingfct : \states \to \powerset{\atomicprops}$
		\end{itemize}
		
		A transition system is called \emph{finite} if \states, \atomicprops and \labelingfct are finite.
	\end{definition}
	
	\redcomment{explanation of components}
	
	\subsection{Markov Chain}
	
	\redcomment{NOTES BEGIN}
	\begin{itemize}
		\item Markov Chain (MC)
		\item transition systems to markov chains: nondeterministic choices replaced by probablistic
		\item successor chosen according to probability distribution
		\item distribution only dependent on current state \state (not path)
		\item system evolution not dependent on history but only current state $\to$ \emph{memoryless property}
	\end{itemize}
	
	\redcomment{NOTES END}
	
	\begin{definition}
		A \emph{(discrete-time) Markov chain} is a tuple $\autm = (\states, \probtfunc, \initdistrib, \atomicprops, \labelingfct)$ where 
		\begin{itemize}
			\item \states is a countable, nonempty set of states,
			\item $\probtfunc : \states \times \states \to [0,1]$ is the \emph{transition probability function}, such that for all states \state:
			\begin{center}
				$\bbigsum{\state' \in \states}\probtfunc(\state,\states') = 1$.	
			\end{center}
			\item $\initdistrib : \states \to [0,1]$ is the \emph{initial distribution}, such that $\mathlarger{\sum_{\state \in \states}} \initdistrib(\state) = 1$, and
			\item \atomicprops is a set of atomic propositions and,
			\item $\labelingfct : \states \to \powerset{\atomicprops}$ a labeling function.		
		\end{itemize}
		
		\autm is called \emph{finite} if \states and \atomicprops are finite. For finite \autm, the \emph{size} of \autm, denoted \emph{size}(\autm),  is the number of states plus the number of pairs $(\state, \state') \in \states \times \states$ with $\probtfunc(\state, \state') > 0.$
	\end{definition}
	
	\redcomment{NOTES BEGIN}
	\begin{itemize}
		\item Probability Function \probtfunc specifies for each state \state the probability \probtfunc(\state,\state') of moving from \state to \state' in one step.
		\item constraint on \probtfunc ensures that \probtfunc is distribution
		\item \initdistrib(\state) specifies system evolution starts in \state
		\item states \state with $\initdistrib(\state) > 0$ are  considered \emph{initial states}
		\item states \state' with $\probtfunc(\state, \state') > 0$ are view as possible successors of \state
		\item has no actions \\ "As compositional approaches for Markov models are outside the scope of this monograph,
		actions are irrelevant in this chapter and are therefore omitted."
	\end{itemize}
	
	\redcomment{NOTES END}
	
	
	\subsection{Markov Decision Process} 
	
	\redcomment{NOTES BEGIN}
	
	\begin{itemize}
		\item Markov decision process (MDP) 
		\item idea: Adding nondeterminism to markov chains. MDPs permit both probabilistic and nondeterministic choices 
		\item probabilistic choices: possible outcomes for of randomized actions -> requires statistical experiments to obtain adequate distributions that model average behavior of the environment
		\item information not available or guarantee about system properties is required -> nondeterminism
		\item Another example: randomized distributed algorithms. Non-determinism: interleaving behavior: nondeterministic choice which process, probabilistic: have rather restricted set of actions that have a random nature
		\item used for abstraction in markov chains: states grouped by \atomicprops and have a wide range of transition probabilities -> essentially nondeterminism -> transition probabilities are replaced by nondeterminism		
	\end{itemize}
	
	\redcomment{NOTES END}
	
	\begin{definition}
		A \emph{Markov decision process} is a tuple $\mdp = \mdptuple$ where
		\begin{itemize}
			\item \states is a countable set of states,
			\item \actions is a set of actions,
			\item $\probtfunc : \states \times \actions \times \states \to [0,1]$ is the transition probability function such that for all states $\state \in \states$ and actions $\action \in \actions$:
			\begin{center}
				$\bbigsum{\state' \in \states}\probtfunc(\state,\action,\state') \in \{0,1\}$,
			\end{center}
			\item $\initdistrib : \states \to [0,1]$ is the initial distribution such that $\sum_{\state \in \states} \initdistrib(\state) = 1$,
			\item \atomicprops is  a set of atomic propositions and
			\item $\labelingfct : \states \to \powerset{\atomicprops}$ a labeling function.
		\end{itemize}
		An action \action is \emph{enabled} in state \state if and only if $\sum_{\state' \in \states} \probtfunc(\state,\action,\state') = 1$. Let $\actions(\state)$ denote the set of enabled actions in \state. For any state $\state \in \states$, it is required that $\actions(\state) \neq \emptyset$. Each state \state' for which $\probtfunc(\state,\action,\state') > 0$ is called an \emph{\action-successor} of \state.
	\end{definition}
	
	An MDP is called \emph{finite} if \states, \actions and \atomicprops are finite.
	\\
	
	\redcomment{NOTES BEGIN}
	
	\begin{itemize}
		\item $\probtfunc(\state, \action, t)$ can be arbitrary real numbers in $[0,1]$ (sum up to 1 or 0 for fixed \state and \action), for algorithmic purposes rational
		\item unique initial distribution \initdistrib. Could be generalized to set of \initdistrib with nondeterministic choice at the beginning. For sake of simplicity: one single distribution
		\item operational behavior:
		\begin{itemize}
			\item starting state $\state_0$ yielded by \initdistrib with $\initdistrib(\state_0) > 0$
			\item nondeterministic choice of enabled action (i.e. Probability sums up to one)
			\item probabilistic choice of state (action fixed by nondeterministic selection)
		\end{itemize}
		
		\item $MC = MDP \iff \forall \state \in \states : |\actions(\state)| = 1$
		
		\item $\implies$ MCs are a proper subset of MDPs
		
	\end{itemize}
	
	\redcomment{NOTES END}
	
	\redcomment{CONVENTION BEGIN}
	
	For $\state_1, \state_2 \in \states$ and $\action \in \actions$ we will write $(\state_1, \action, \state_2) \in \probtfunc$ \iffN $\probtfunc(\state_1, \action, \state_2) > 0$, that is to say if there is a non-zero probability of moving from state $\state_1$ to state $\state_2$ with action \action. Analogously we will write $\state \in \initdistrib$ \iffN $\initdistrib(\state) > 0$ which means that there is chance of starting in that state.
	
	\redcomment{not sure if I should: $I := \initdistrib$ thereby meaning the underlying set}
	
	\redcomment{CONVENTION END TEST}
	
\end{document}
