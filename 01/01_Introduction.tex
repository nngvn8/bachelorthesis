\documentclass[preview]{standalone}
%\usepackage{prelude}

%%%%%%%%%%%%%%%%%%%%%%%%%%%%%%%%%%%% PACKAGES %%%%%%%%%%%%%%%%%%%%%%%%%%%%%%%%%%%%%%%%%%

\usepackage{inputenc,fontenc}
\usepackage[a4paper,margin=3cm]{geometry}
\usepackage[english,german]{babel}
%\usepackage[german]{babel}
%\usepackage[fixlanguage]{babelbib}


\usepackage{bbold}
\usepackage{amsthm}
\usepackage{amsmath}
\usepackage[dvipsnames]{xcolor}
\usepackage{standalone}
\usepackage{tikz}
\usepackage{cite}
\usepackage{xspace}
\usepackage{relsize}
\usepackage{mathtools} %mathclap
%\usepackage{xparse} % \newDocumentCommand for multiple optional arguments
%\usepackage{titlecaps}


%%%%%%%%%%%%%%%%%%%%%%%%%%%%%%%%%%%% THEOREMSTYLES %%%%%%%%%%%%%%%%%%%%%%%%%%%%%%%%%%

\theoremstyle{definition}
\newtheorem{definition}{Definition}[section]

\theoremstyle{theorem}
\newtheorem{theorem}{Satz}[section]

\theoremstyle{korollary}
\newtheorem{korollary}{Korollar}[section]

\theoremstyle{definition}
\newtheorem{exmp}{Beispiel}[section]


%%%%%%%%%%%%%%%%%%%%%%%%%%%%%%%%%%% MY MACROS %%%%%%%%%%%%%%%%%%%%%%%%%%%%%%%%%%%%%%%%%
%formatting
\newcommand{\comment}[2]{{\color{#1}#2}}
\newcommand{\redcomment}[1]{{\color{red}#1}}
\newcommand{\purpcomment}[1]{{\color{pink}#1}}
\newcommand{\bluecomment}[1]{{\color{blue}#1}}
\newcommand{\mt}[1]{\ensuremath{{#1}}\xspace}
\newcommand{\mynewcommand}[2]{\newcommand{#1}{\mt{#2}}} %% currently not used becaue of ide highlighting

%own names
\newcommand{\nm}[1]{#1\xspace}
\newcommand{\viewN}{\nm{view}}
\newcommand{\viewNC}{\nm{View}}
\newcommand{\viewsN}{\nm{views}}
\newcommand{\viewsNC}{\nm{Views}}
\newcommand{\grpfctN}{\nm{grouping function}}
\newcommand{\grpfctNC}{\nm{Grouping function}}
\newcommand{\grpfctNCC}{\nm{Grouping Function}}
\newcommand{\grpfctsN}{\nm{grouping functions}}
\newcommand{\grpfctsNC}{\nm{Grouping functions}}
\newcommand{\grpfctsNCC}{\nm{Grouping Functions}}
\newcommand{\chosengraphtypeNCC}{\nm{Transition System}}
\newcommand{\chosengraphtypeNC}{\nm{Transition system}}
\newcommand{\chosengraphtypeN}{\nm{transition system}}
\newcommand{\chosengraphtypesNCC}{\nm{Transition Systems}}
\newcommand{\chosengraphtypesNC}{\nm{Transition systems}}
\newcommand{\chosengraphtypesN}{\nm{transition systems}}
\newcommand{\parllcompN}{\nm{parallel composition}}
\newcommand{\parllcompNC}{\nm{Parallel composition}}
\newcommand{\parllcompNCC}{\nm{Parallel Composition}}

\newcommand{\outactident}{\nm{OutActionsIdent}}

%names
\newcommand{\iffN}{\nm{if and only if}}
\newcommand{\tsN}{\nm{TS}}

%% outactions identical
\newcommand{\outactidentstrong}{\nm{strong}}
\newcommand{\outactidentweak}{\nm{weak}}

%Identifiers for self created definitions
\newcommand{\grpfct}{\mt{F}}
\newcommand{\view}[2][\viewid]{\mt{#1_#2}}
\newcommand{\imggrp}{\mt{\arbset}}
\newcommand{\viewppty}{\mt{\grpfct}}
\newcommand{\viewid}{\mt{\ts}}
\newcommand{\pll}{\mt{||}}
\newcommand{\remstates}{\mt{\bigcup_{\state \in \states \setminus \states_1}\{\state\}}}


%example views
\newcommand{\outact}{\mt{\overrightarrow{\action}}}
\newcommand{\outacts}{\mt{\overrightarrow{\actions}}}
\newcommand{\inact}{\mt{\overleftarrow{\action}}}
\newcommand{\inacts}{\mt{\overleftarrow{\actions(\state)}}}
\newcommand{\gfctatomicprops}{\mt{{\viewppty_\atomicprops}}}
\newcommand{\gfctinitstates}{\mt{{\viewppty_\initstates}}}
\newcommand{\gfcthasoutaction}{\mt{{\viewppty_{\exists\outact}}}}
\newcommand{\gfctminoutaction}{\mt{{\viewppty_{\numoutact\leq\outact}}}}
\newcommand{\gfctmaxoutaction}{\mt{{\viewppty_{\outact\leq\numoutact}}}}
\newcommand{\gfctspanoutaction}{\mt{{\viewppty_{\numoutactb\leq\outact\leq\numoutact}}}}
\newcommand{\gfctstrongoutactident}{\mt{{\viewppty_{\outacts(\state)_=}}}}
\newcommand{\gfctweakoutactident}{\mt{{\viewppty_{\outacts(\state)_\approx}}}}
\newcommand{\gfcthasinaction}{\mt{{\viewppty_{\exists\inact}}}}
\newcommand{\gfctmininaction}{\mt{{\viewppty_{\numinact\leq\inact}}}}
\newcommand{\gfctmaxinaction}{\mt{{\viewppty_{\inact\leq\numinact}}}}
\newcommand{\gfctspaninaction}{\mt{{\viewppty_{\numinactb\leq\inact\leq\numinact}}}}
\newcommand{\gfctstronginactident}{\mt{{\viewppty_{\inacts(\state)_=}}}}
\newcommand{\gfctweakinactident}{\mt{{\viewppty_{\intacts(\state)_\approx}}}}
\newcommand{\gfctparamvalueseq}{\mt{\viewppty_{\param = \paramval}}}
\newcommand{\gfctparamvaluesneq}{\mt{\viewppty_{\param \neq \paramval}}}
\newcommand{\gfctparamvalueseqopt}[1][\paramval]{\mt{\viewppty_{\param = #1}}}
\newcommand{\gfctparamvalident}{\mt{\viewppty_{\parameval(\state,\param)}}}


\newcommand{\viewatomicprops}{\mt{\view{\gfctatomicprops}}}
\newcommand{\viewinitstates}{\mt{\view{\gfctinitstates}}}
\newcommand{\viewhasoutaction}{\mt{\view{\gfcthasoutaction}}}
\newcommand{\viewminoutaction}{\mt{\view{\gfctminoutaction}}}
\newcommand{\viewmaxoutaction}{\mt{\view{\gfctmaxoutaction}}}
\newcommand{\viewspanoutaction}{\mt{\view{\gfctspanoutaction}}}
\newcommand{\viewstrongoutactident}{\mt{\view{\gfctstrongoutactident}}}
\newcommand{\viewweakoutactident}{\mt{\view{\gfctweakoutactident}}}
\newcommand{\viewhasinaction}{\mt{\view{\gfcthasinaction}}}
\newcommand{\viewmininaction}{\mt{\view{\gfctmininaction}}}
\newcommand{\viewmaxinaction}{\mt{\view{\gfctmaxinaction}}}
\newcommand{\viewspaninaction}{\mt{\view{\gfctspaninaction}}}
\newcommand{\viewstronginactident}{\mt{\view{\gfctstronginactident}}}
\newcommand{\viewweakinactident}{\mt{\view{\gfctweakinactident}}}
\newcommand{\viewparamvalueseq}{\mt{\view{\gfctparamvalueseq}}}
\newcommand{\viewparamvaluesneq}{\mt{\view{\gfctparamvaluesneq}}}
\newcommand{\viewparamvalident}{\mt{\view{\gfctparamvalident}}}

%%OutAct
\newcommand{\numoutact}{\mt{n}}
\newcommand{\numoutactb}{\mt{m}}
\newcommand{\numinact}{\mt{n}}
\newcommand{\numinactb}{\mt{m}}
\newcommand{\setoutact}{\mt{\actions \cup \states}}

\newcommand{\predmaxoutact}[1][\numoutact]{\mt{Q_{\outact\leq#1}(\state,\state_1, \dots, \state_{#1+1})}}
\newcommand{\predminoutact}[1][\numoutact]{\mt{Q_{#1\leq\outact}(\state,\state_1, \dots, \state_{#1})}}
\newcommand{\formoutact}[1][\state]{\mt{C_{#1,\outact}}}
\newcommand{\predmaxinact}[1][\numinact]{\mt{Q_{\inact\leq#1}(\state,\state_1, \dots, \state_{#1+1})}}
\newcommand{\predmininact}[1][\numinact]{\mt{Q_{#1\leq\inact}(\state,\state_1, \dots, \state_{#1})}}

%%Parameters
\newcommand{\params}{\mt{Par}}
\newcommand{\param}{\mt{x}}
\newcommand{\paramval}{\mt{a}}
\newcommand{\parameval}{\mt{ParEval}}
\newcommand{\paramevalimg}{\mt{\arbset}}
\newcommand{\someparam}{\mt{\tilde{x}}}

%Transitionsystem
\newcommand{\ts}{\mt{TS}}
\newcommand{\state}{\mt{s}}
\newcommand{\action}{\mt{\alpha}}
\newcommand{\actionb}{\mt{\beta}}
\newcommand{\actionc}{\mt{\alpha}}
\newcommand{\atomicprop}{\mt{ap}}

\newcommand{\smstate}{\mt{\tilde{\state}}}

\newcommand{\states}{\mt{S}}
\newcommand{\actions}{\mt{Act}}
\newcommand{\transitionrel}{\mt{\longrightarrow}}
\newcommand{\initstates}{\mt{I}}
\newcommand{\atomicprops}{\mt{AP}}
\newcommand{\labelingfct}{\mt{L}}
\newcommand{\transitionsystem}{\mt
	{(\states, \actions, \transitionrel, \initstates, \atomicprops, \labelingfct)}
}
\newcommand{\eqrelview}{\mt{R}}
\newcommand{\eqclassv}[1]{\mt{\eqclass{#1}{\eqrelview}}}

%Markov chains and MDP
\newcommand{\autm}{\mt{\mathcal{M}}}
\newcommand{\probtfunc}{\mt{\textbf{P}}}
\newcommand{\initdist}{\redcomment{\mt{l_{init}}}}



%maths
\newcommand{\powerset}[1]{\mt{\mathcal{P}(#1)}}
\newcommand{\eqclass}[2]{\mt{[#1]_{#2}}}%{\mt{#1 / #2}}
\newcommand{\impr}{\mt{\hspace{3mm}\Rightarrow\hspace{2mm}}}
\newcommand{\impl}{\mt{\hspace{3mm}\Leftarrow\hspace{2mm}}}
\newcommand{\natnums}{\mt{\mathbb{N}}} 
\newcommand{\arbset}{\mt{M}}
\newcommand{\bigsum}[2][]{\mt{\mathlarger{\sum}_{#2}^{#1}}}
\newcommand{\bbigsum}[2][]{\mt{\mathlarger{\mathlarger{\sum}}_{#2}^{#1}}}
\newcommand{\invimage}[2]{#1^{\mt{-1}(#2)}}

%tickz
%% \definecolor{darkred}{RGB}{196, 42, 42}


\begin{document}
\section{Introduction}
	
The modern world heavily relies on ICT (Information and Communication Technology) systems. They are found in devices used everyday like smartphones or laptops or in distributed systems such as the infrastructure sustaining the internet, but also in life-saving ones utilized in medicine. Apart from performance, provided features and functionality, one of the most relevant aspects of such systems is their fault free behavior during runtime. The effect of a system fault can reach from a small disturbance in the user experience to inoperability of the whole system, the effect of which could be an annoyed user, a financial damage of several millions or a lost live. To prevent these possibly severe negative effects, methods to verify the correct behavior of a system are needed.

Apart from in practice often used approaches like peer review and testing for software and emulation and simulation for hardware, formal methods can be used as well to verify the correct behavior of a system. 
These can be based on models, where a state in the model refers to a possible state of the system. Models describe the possible system behavior in a mathematically precise and unambiguous manner and can be utilized in various ways for verification: the state space can be explored exhaustively, only specific scenarios can be considered or they are tested in reality.

Model checking describes the approach of exhaustive exploration of the state space. It is a formal model-based method for system verification, that checks in an automated manner if a given property holds for every state of a given finite-state model. \cite[chs. 1.1 and 1.2]{Baier2008}. Probabilistic model checking allows not only nondeterministic transitions between states in the models checked, but also probabilistic ones. It enables to properly model and check systems in which both controllable, as well as stochastic phenomena are occurring. The major limitation for algorithms running on the set of states of models, in order to model check them, is the state explosion problem. The state explosion problem states that the number of states grow exponentially in the number of variables in a program graph or the number of components in concurrent systems \cite[ch. 2.3]{Baier2008}. Already simple system models can lead to complex system behavior and an immense amount of states. This is not only a problem for algorithms but also humans who need to understand, analyze and review models in the context of model checking. An interactive visualization can assist with this issue and even be made use of to achieve further goals such as debugging or model repair. 
 
There are techniques in visualization for large multivariate graphs (networks where nodes and relationships have attributes), such as \mdpsN. Aggregation and clustering methods are used in visualization for large graphs\cite{Goerke2009}. There also exist methods for visualizing multivariate graphs \cite{Kerren2014,Nobre2019}. One approach are multiple coordinated view setups featuring parallel coordinates plots (PCPs) \cite{Johansson2016}. But even these techniques have their limitations for very large amounts of data as they are easily reachable with models used in model checking. The pure data volume is left unaffected. One could address this issue by reducing the amount of data, passed on to the visualization. This is what this thesis aims for: preprocessing the data, that is then utilized in visualization.

%In this thesis we want to preprocess data, that is then utilized in visualization. 
Preprocessing is a term that describes an approach mostly used in Data Mining and Machine Learning. Common problems are too much data, too little data and imperfect data (noise, incompleteness)\cite{Garcia2016}. Preprocessing addresses these issues. As the state explosion problem causes an immense amount of states, approaches that reduce the amount of data are of interest. Methods reducing data can refer to size of single samples (feature selection, space transformations) or the amount of samples (instance reduction). There are several methods to reduce the amount of instances. The approach that will be explored in this thesis can be classified as instance reduction and can be considered as a variant of clustering, although in literature clustering can describe slightly different notions \cite{Alasadi2017,Baskar2013}.

Whereas these approaches are general in the sense of that they concern arbitrary datatypes, the representations of \mdpsN are graphs, which have been heavily studied in the last decades. In consequence there exist many approaches to simplify graphs. There are simplifications based on the pure mathematical object, without any domain specific context such as connectivity \cite{Zhou2010}, patterns\cite{Soldano2014}, modularity \cite{Arenas2007} or cuts \cite{Goerke2009,Fung2011}. There also exist methods specific to certain domains \cite{Ruan2011,Li2022,Yaw2019}. In this thesis we want to explore approaches specific to the domain of model checking. Approaches found in literature aim for preserving certain logic formulae or the ability to perform proper verification on the simplified graph \cite{Rensink2012,Bonchi2013,Boneva2007}. An interesting model checking specific approach is based on equivalence or order relations on the set of states. States are in relation, if they can simulate the other stepwise or in several steps with respect to atomic propositions. This causes preservation of certain logical formulae used in model checking, but conversely other information is lost. In addition the computation of these relations is rather costly. We will introduce and provide an implementation of a concept, which will be called \viewN, that is similar to abstraction \cite[pp. 499]{Baier2008}. It defines an equivalence relation on the set of states which share certain traits. Its intention is to show humans interacting with the visualization as much information possible in compact and concise form, rather than preserving logical formulae.

After giving some fundamentals about the systems where the concept \viewN may be applied in chapter \ref{ch:prelim} we will formalize \viewsN, discuss types and operations on and with them in chapter \ref{ch:view}. In chapter \ref{ch:viewexmp} we will give examples of views utilizing \chgphN characteristics and \viewsN utilizing the structure of the \chgphN graph, but for this bachelor thesis limit the scope of views considered to those, that do not take advantage of probabilities in \chgphsN. In chapter \ref{ch:viewimpl} we will elaborate on where and how the proposed \viewsN of chapter \ref{ch:viewexmp} have been implemented. Lastly the \viewsN will be evaluated in chapter \ref{ch:eval} by considering three use cases how \viewsN can be applied and used. Moreover there will be an overview about performance and scalability of the proposed \viewsN. 

%\begin{itemize}
%	
%	\item Fung sparsification - no domain specific
%	\item Zhung connectivity
%	\item Soldano Patterns topological constraint in Graphs - no domain specific
%	\item Arena - module
%	\item Ruansen Maintain shortest path
%	\item Yaw2019 simplifcation for networks
%	\item Li2022 Dyck reachability
%	\item Bonchi2013 activity traces + general appraoches
%	\item Boneva edge labeling, neighbourhood,  preservance of formulae
%	\item Resink Verification of infinite graphs on abstract level using patterns
%	
%\end{itemize}

\end{document}
