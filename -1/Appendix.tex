\documentclass[preview]{standalone}

%%%%%%%%%%%%%%%%%%%%%%%%%%%%%%%%%%%% PACKAGES %%%%%%%%%%%%%%%%%%%%%%%%%%%%%%%%%%%%%%%%%%

\usepackage{inputenc,fontenc}
\usepackage[a4paper,margin=3cm]{geometry}
\usepackage[english,german]{babel}
%\usepackage[german]{babel}
%\usepackage[fixlanguage]{babelbib}


\usepackage{bbold}
\usepackage{amsthm}
\usepackage{amsmath}
\usepackage[dvipsnames]{xcolor}
\usepackage{standalone}
\usepackage{tikz}
\usepackage{cite}
\usepackage{xspace}
\usepackage{relsize}
\usepackage{mathtools} %mathclap
%\usepackage{xparse} % \newDocumentCommand for multiple optional arguments
%\usepackage{titlecaps}


%%%%%%%%%%%%%%%%%%%%%%%%%%%%%%%%%%%% THEOREMSTYLES %%%%%%%%%%%%%%%%%%%%%%%%%%%%%%%%%%

\theoremstyle{definition}
\newtheorem{definition}{Definition}[section]

\theoremstyle{theorem}
\newtheorem{theorem}{Satz}[section]

\theoremstyle{korollary}
\newtheorem{korollary}{Korollar}[section]

\theoremstyle{definition}
\newtheorem{exmp}{Beispiel}[section]


%%%%%%%%%%%%%%%%%%%%%%%%%%%%%%%%%%% MY MACROS %%%%%%%%%%%%%%%%%%%%%%%%%%%%%%%%%%%%%%%%%
%formatting
\newcommand{\comment}[2]{{\color{#1}#2}}
\newcommand{\redcomment}[1]{{\color{red}#1}}
\newcommand{\purpcomment}[1]{{\color{pink}#1}}
\newcommand{\bluecomment}[1]{{\color{blue}#1}}
\newcommand{\mt}[1]{\ensuremath{{#1}}\xspace}
\newcommand{\mynewcommand}[2]{\newcommand{#1}{\mt{#2}}} %% currently not used becaue of ide highlighting

%own names
\newcommand{\nm}[1]{#1\xspace}
\newcommand{\viewN}{\nm{view}}
\newcommand{\viewNC}{\nm{View}}
\newcommand{\viewsN}{\nm{views}}
\newcommand{\viewsNC}{\nm{Views}}
\newcommand{\grpfctN}{\nm{grouping function}}
\newcommand{\grpfctNC}{\nm{Grouping function}}
\newcommand{\grpfctNCC}{\nm{Grouping Function}}
\newcommand{\grpfctsN}{\nm{grouping functions}}
\newcommand{\grpfctsNC}{\nm{Grouping functions}}
\newcommand{\grpfctsNCC}{\nm{Grouping Functions}}
\newcommand{\chosengraphtypeNCC}{\nm{Transition System}}
\newcommand{\chosengraphtypeNC}{\nm{Transition system}}
\newcommand{\chosengraphtypeN}{\nm{transition system}}
\newcommand{\chosengraphtypesNCC}{\nm{Transition Systems}}
\newcommand{\chosengraphtypesNC}{\nm{Transition systems}}
\newcommand{\chosengraphtypesN}{\nm{transition systems}}
\newcommand{\parllcompN}{\nm{parallel composition}}
\newcommand{\parllcompNC}{\nm{Parallel composition}}
\newcommand{\parllcompNCC}{\nm{Parallel Composition}}

\newcommand{\outactident}{\nm{OutActionsIdent}}

%names
\newcommand{\iffN}{\nm{if and only if}}
\newcommand{\tsN}{\nm{TS}}

%% outactions identical
\newcommand{\outactidentstrong}{\nm{strong}}
\newcommand{\outactidentweak}{\nm{weak}}

%Identifiers for self created definitions
\newcommand{\grpfct}{\mt{F}}
\newcommand{\view}[2][\viewid]{\mt{#1_#2}}
\newcommand{\imggrp}{\mt{\arbset}}
\newcommand{\viewppty}{\mt{\grpfct}}
\newcommand{\viewid}{\mt{\ts}}
\newcommand{\pll}{\mt{||}}
\newcommand{\remstates}{\mt{\bigcup_{\state \in \states \setminus \states_1}\{\state\}}}


%example views
\newcommand{\outact}{\mt{\overrightarrow{\action}}}
\newcommand{\outacts}{\mt{\overrightarrow{\actions}}}
\newcommand{\inact}{\mt{\overleftarrow{\action}}}
\newcommand{\inacts}{\mt{\overleftarrow{\actions(\state)}}}
\newcommand{\gfctatomicprops}{\mt{{\viewppty_\atomicprops}}}
\newcommand{\gfctinitstates}{\mt{{\viewppty_\initstates}}}
\newcommand{\gfcthasoutaction}{\mt{{\viewppty_{\exists\outact}}}}
\newcommand{\gfctminoutaction}{\mt{{\viewppty_{\numoutact\leq\outact}}}}
\newcommand{\gfctmaxoutaction}{\mt{{\viewppty_{\outact\leq\numoutact}}}}
\newcommand{\gfctspanoutaction}{\mt{{\viewppty_{\numoutactb\leq\outact\leq\numoutact}}}}
\newcommand{\gfctstrongoutactident}{\mt{{\viewppty_{\outacts(\state)_=}}}}
\newcommand{\gfctweakoutactident}{\mt{{\viewppty_{\outacts(\state)_\approx}}}}
\newcommand{\gfcthasinaction}{\mt{{\viewppty_{\exists\inact}}}}
\newcommand{\gfctmininaction}{\mt{{\viewppty_{\numinact\leq\inact}}}}
\newcommand{\gfctmaxinaction}{\mt{{\viewppty_{\inact\leq\numinact}}}}
\newcommand{\gfctspaninaction}{\mt{{\viewppty_{\numinactb\leq\inact\leq\numinact}}}}
\newcommand{\gfctstronginactident}{\mt{{\viewppty_{\inacts(\state)_=}}}}
\newcommand{\gfctweakinactident}{\mt{{\viewppty_{\intacts(\state)_\approx}}}}
\newcommand{\gfctparamvalueseq}{\mt{\viewppty_{\param = \paramval}}}
\newcommand{\gfctparamvaluesneq}{\mt{\viewppty_{\param \neq \paramval}}}
\newcommand{\gfctparamvalueseqopt}[1][\paramval]{\mt{\viewppty_{\param = #1}}}
\newcommand{\gfctparamvalident}{\mt{\viewppty_{\parameval(\state,\param)}}}


\newcommand{\viewatomicprops}{\mt{\view{\gfctatomicprops}}}
\newcommand{\viewinitstates}{\mt{\view{\gfctinitstates}}}
\newcommand{\viewhasoutaction}{\mt{\view{\gfcthasoutaction}}}
\newcommand{\viewminoutaction}{\mt{\view{\gfctminoutaction}}}
\newcommand{\viewmaxoutaction}{\mt{\view{\gfctmaxoutaction}}}
\newcommand{\viewspanoutaction}{\mt{\view{\gfctspanoutaction}}}
\newcommand{\viewstrongoutactident}{\mt{\view{\gfctstrongoutactident}}}
\newcommand{\viewweakoutactident}{\mt{\view{\gfctweakoutactident}}}
\newcommand{\viewhasinaction}{\mt{\view{\gfcthasinaction}}}
\newcommand{\viewmininaction}{\mt{\view{\gfctmininaction}}}
\newcommand{\viewmaxinaction}{\mt{\view{\gfctmaxinaction}}}
\newcommand{\viewspaninaction}{\mt{\view{\gfctspaninaction}}}
\newcommand{\viewstronginactident}{\mt{\view{\gfctstronginactident}}}
\newcommand{\viewweakinactident}{\mt{\view{\gfctweakinactident}}}
\newcommand{\viewparamvalueseq}{\mt{\view{\gfctparamvalueseq}}}
\newcommand{\viewparamvaluesneq}{\mt{\view{\gfctparamvaluesneq}}}
\newcommand{\viewparamvalident}{\mt{\view{\gfctparamvalident}}}

%%OutAct
\newcommand{\numoutact}{\mt{n}}
\newcommand{\numoutactb}{\mt{m}}
\newcommand{\numinact}{\mt{n}}
\newcommand{\numinactb}{\mt{m}}
\newcommand{\setoutact}{\mt{\actions \cup \states}}

\newcommand{\predmaxoutact}[1][\numoutact]{\mt{Q_{\outact\leq#1}(\state,\state_1, \dots, \state_{#1+1})}}
\newcommand{\predminoutact}[1][\numoutact]{\mt{Q_{#1\leq\outact}(\state,\state_1, \dots, \state_{#1})}}
\newcommand{\formoutact}[1][\state]{\mt{C_{#1,\outact}}}
\newcommand{\predmaxinact}[1][\numinact]{\mt{Q_{\inact\leq#1}(\state,\state_1, \dots, \state_{#1+1})}}
\newcommand{\predmininact}[1][\numinact]{\mt{Q_{#1\leq\inact}(\state,\state_1, \dots, \state_{#1})}}

%%Parameters
\newcommand{\params}{\mt{Par}}
\newcommand{\param}{\mt{x}}
\newcommand{\paramval}{\mt{a}}
\newcommand{\parameval}{\mt{ParEval}}
\newcommand{\paramevalimg}{\mt{\arbset}}
\newcommand{\someparam}{\mt{\tilde{x}}}

%Transitionsystem
\newcommand{\ts}{\mt{TS}}
\newcommand{\state}{\mt{s}}
\newcommand{\action}{\mt{\alpha}}
\newcommand{\actionb}{\mt{\beta}}
\newcommand{\actionc}{\mt{\alpha}}
\newcommand{\atomicprop}{\mt{ap}}

\newcommand{\smstate}{\mt{\tilde{\state}}}

\newcommand{\states}{\mt{S}}
\newcommand{\actions}{\mt{Act}}
\newcommand{\transitionrel}{\mt{\longrightarrow}}
\newcommand{\initstates}{\mt{I}}
\newcommand{\atomicprops}{\mt{AP}}
\newcommand{\labelingfct}{\mt{L}}
\newcommand{\transitionsystem}{\mt
	{(\states, \actions, \transitionrel, \initstates, \atomicprops, \labelingfct)}
}
\newcommand{\eqrelview}{\mt{R}}
\newcommand{\eqclassv}[1]{\mt{\eqclass{#1}{\eqrelview}}}

%Markov chains and MDP
\newcommand{\autm}{\mt{\mathcal{M}}}
\newcommand{\probtfunc}{\mt{\textbf{P}}}
\newcommand{\initdist}{\redcomment{\mt{l_{init}}}}



%maths
\newcommand{\powerset}[1]{\mt{\mathcal{P}(#1)}}
\newcommand{\eqclass}[2]{\mt{[#1]_{#2}}}%{\mt{#1 / #2}}
\newcommand{\impr}{\mt{\hspace{3mm}\Rightarrow\hspace{2mm}}}
\newcommand{\impl}{\mt{\hspace{3mm}\Leftarrow\hspace{2mm}}}
\newcommand{\natnums}{\mt{\mathbb{N}}} 
\newcommand{\arbset}{\mt{M}}
\newcommand{\bigsum}[2][]{\mt{\mathlarger{\sum}_{#2}^{#1}}}
\newcommand{\bbigsum}[2][]{\mt{\mathlarger{\mathlarger{\sum}}_{#2}^{#1}}}
\newcommand{\invimage}[2]{#1^{\mt{-1}(#2)}}

%tickz
%% \definecolor{darkred}{RGB}{196, 42, 42}

\begin{document}
\section{Appendix}
\subsection[]{Java Code of Distance Views} \label{apx:distancejava}



\begin{lstlisting}[style=javaStyle, caption={grp Function java}]
protected List<String> groupingFunction() throws Exception {
	List<String> toExecute = new ArrayList<>();
	
	// Compute reachability score
	Set<Long> visited = new HashSet<>();
	Set<Long> visiting = new HashSet<>();
	long distance = 0;
	
	// initialise states with expression
	if (identifierExpression.equals("init")){
		Set<Long> subsetInitStates = model.getInitialStates()
		.stream()
		.filter(stateId -> relevantStates.contains(stateId))
		.collect(Collectors.toSet());
		visiting.addAll(subsetInitStates);
	} else {
		Set<Long> subsetStates = model.getStatesByExpression(model.parseSingleExpressionString(identifierExpression).toString())
		.stream()
		.filter(stateId -> relevantStates.contains(stateId))
		.collect(Collectors.toSet());
		visiting.addAll(subsetStates);
	}
	MdpGraph mdpGraph = model.getMdpGraph();

// BFS: determine distance from states given by (PRISM) expression
while(!visiting.isEmpty()){
	Set<Long> toVisit = new HashSet<>();
	
	for (Long stateID : visiting){
		long curr = distance - Math.floorMod(distance, granularity);
		
		// create SQL statement
		toExecute.add(String.format("UPDATE %s SET %s = '%s' WHERE %s = '%s'", model.getStateTableName(), getCollumn(), curr, ENTRY_S_ID, stateID));
		
		Set<Long> reachableStates;
		
		// see all outgoing (depending Direction Mode)
		switch (direction) {
			case BACKWARD:
			reachableStates = mdpGraph.incomingEdgesOf(stateID)
			.stream()
			.map(mdpGraph::getEdgeSource)
			.filter(stateId -> relevantStates.contains(stateId))
			.collect(Collectors.toSet());
			break;
						
			case DIRECTIONLESS:
			reachableStates =
			mdpGraph.outgoingEdgesOf(stateID)
			.stream()
			.map(mdpGraph::getEdgeTarget)
			.filter(stateId -> relevantStates.contains(stateId))
			.collect(Collectors.toSet());
			reachableStates.addAll(
			mdpGraph.incomingEdgesOf(stateID)
			.stream()
			.map(mdpGraph::getEdgeSource)
			.filter(stateId -> relevantStates.contains(stateId))
			.collect(Collectors.toSet())
			);
			break;
			
			case FORWARD: default:
			reachableStates = mdpGraph.outgoingEdgesOf(stateID)
			.stream()
			.map(mdpGraph::getEdgeTarget)
			.filter(stateId -> relevantStates.contains(stateId))
			.collect(Collectors.toSet());
		}
		
		for (Long idReachableState : reachableStates) {
			if (!(visited.contains(idReachableState) ||
			   visiting.contains(idReachableState))){
				toVisit.add(idReachableState);
			}
		}
	}
	
	visited.addAll(visiting);
	visiting = toVisit;
	distance++;
	}
	
	// compute not reachable states
	Set<Long> not_reachable = new HashSet<>(model.getAllStates())
	.stream()
	.filter(stateId -> relevantStates.contains(stateId))
	.collect(Collectors.toSet());
	not_reachable.removeAll(visited);
	
	// compute SQL statements for not reachable states
	for (long stateID : not_reachable){
		String reachability = semiGrouping ? ENTRY_C_BLANK : "inf";
		toExecute.add(String.format("UPDATE %s SET %s = '%s' WHERE %s = '%s'", model.getStateTableName(), getCollumn(), reachability, ENTRY_S_ID, stateID));
	}
	return toExecute;
}
\end{lstlisting}


\subsection[\prism-File Dining Philosophers]{\prism-File for \mdpN of Dining Philosophers} \label{lst:0601diningphilosophersfull}
%\begin{figure}
\begin{lstlisting}[language=prism, caption={PRISM model file for \mdpN of Dining Philosophers}]
	formula lfree = (p2>=0&p2<=4)|p2=6|p2=10;
	formula rfree = (p3>=0&p3<=3)|p3=5|p3=7|p3=11;
	
	module phil1
	
	p1: [0..11];
	
	[] p1=0 -> (p1'=0); // stay thinking
	[] p1=0 -> (p1'=1); // trying
	[] p1=1 -> 0.5 : (p1'=2) + 0.5 : (p1'=3); // draw randomly
	[] p1=2 &  lfree -> (p1'=4); // pick up left
	[] p1=2 &  !lfree -> (p1'=2); // left not free
	[] p1=3 &  rfree -> (p1'=5); // pick up right
	[] p1=3 &  !rfree -> (p1'=3); // right not free
	[] p1=4 &  rfree -> (p1'=8); // pick up right (got left)
	[] p1=4 & !rfree -> (p1'=6); // right not free (got left)
	[] p1=5 &  lfree -> (p1'=8); // pick up left (got right)
	[] p1=5 & !lfree -> (p1'=7); // left not free (got right)
	[] p1=6  -> (p1'=1); // put down left
	[] p1=7  -> (p1'=1); // put down right
	[] p1=8  -> (p1'=9); // move to eating (got forks)
	[] p1=9  -> (p1'=10); // finished eating and put down left 
	[] p1=9  -> (p1'=11); // finished eating and put down right
	[] p1=10 -> (p1'=0); // put down right and return to think
	[] p1=11 -> (p1'=0); // put down left and return to think
	
	endmodule
	
	// construct further modules through renaming
	module phil2 = phil1 [ p1=p2, p2=p3, p3=p1 ] endmodule
	module phil3 = phil1 [ p1=p3, p2=p1, p3=p2 ] endmodule	
			
\end{lstlisting}
%\end{figure}
	
%	// labels
%	label "hungry" = ((p1>0)&(p1<8))|((p2>0)&(p2<8))|((p3>0)&(p3<8));
%	label "eat" = ((p1>=8)&(p1<=9))|((p2>=8)&(p2<=9))|((p3>=8)&(p3<=9));
%	// randomized dining philosophers [LR81]
%// dxp/gxn 12/12/99
%// atomic formulae 
%// left fork free and right fork free resp.
\pagebreak

\subsection[Figures Large]{Large Figures from Figure \ref{fig:0603timeandphases} in Chapter \ref{ch:eval}}

\begin{figure}[!htb]
	\includegraphics[width=\textwidth]{./06/images/06_03_time.png}
\end{figure}

\begin{figure}[!htb]
	\includegraphics[width=\textwidth]{./06/images/06_03_phases.png}
\end{figure}

\pagebreak

\subsection[Classes Time Measuring and Usage Example]{Timer Class and TimeSaver Class with Usage Example}\label{apx:timerandexmp}
\begin{figure}[!htb]
\begin{lstlisting}[style=javaStyle, caption={Timer.java}]
package prism.misc;

import java.time.LocalTime;

public class Timer implements AutoCloseable {
	TimeSaver timeSaver;
	
	public Timer(TimeSaver timeSaver){
		this.timeSaver = timeSaver;
		timeSaver.storeStartTime(LocalTime.now());
	}
	
	@Override
	public void close() {
		timeSaver.storeEndTime(LocalTime.now());
	}
}

}
\end{lstlisting}
\end{figure}

\begin{figure}[!htb]
	\begin{lstlisting}[style=javaStyle, caption={Example for measuring time with Timer}]
		import prism.misc.Timer;
		import prism.misc.TimeSaver;
		
		class Testtimer {
			public static void main(String[] args) {				
				TimeSaver timeSaver = new TimeSaver();				
				for (int i = 0; i < 100; i++) {
					try(Timer timer = new Timer(timeSaver)) {
						testFunction();
					}
				}			
				System.out.println(timeSaver.getAverageDuration());				
			}
			
			private void testFunction(){
				// do something
			}
		}
	\end{lstlisting}
\end{figure}

\begin{figure}
\begin{lstlisting}[style=javaStyle, caption={TimeSaver.java}]		
package prism.misc;

import java.util.*;
import java.time.LocalTime;
import java.time.Duration;

public class TimeSaver {
	
	private List<LocalTime> startTimes = new ArrayList<>();
	
	private List<LocalTime> endTimes = new ArrayList<>();
	
	public double getAvgDurationInMs() {
		long amountSamples;
		if (startTimes.size() != endTimes.size()) {
			throw new RuntimeException("startTimes.size() != endTimes.size()");
		} else {
			amountSamples = startTimes.size();
		}
		long durationAccLong = 0;
		for (int i = 0; i < amountSamples; i++) {
			durationAccLong += getDurationInMsAt(i);
		}
		return ((double)durationAccLong) / ((double)amountSamples);
	}
	
	public long getDurationInMsAt(int i) {
		return Duration.between(startTimes.get(i), endTimes.get(i)).toMillis();
	}
	
	public void storeStartTime(LocalTime startTime) {
		this.startTimes.add(startTime);
	}
	
	public void storeEndTime(LocalTime endTime) {
		this.endTimes.add(endTime);
	}

}
\end{lstlisting}
\end{figure}


\end{document}