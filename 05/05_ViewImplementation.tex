\documentclass[preview]{standalone}

%%%%%%%%%%%%%%%%%%%%%%%%%%%%%%%%%%%% PACKAGES %%%%%%%%%%%%%%%%%%%%%%%%%%%%%%%%%%%%%%%%%%

\usepackage{inputenc,fontenc}
\usepackage[a4paper,margin=3cm]{geometry}
\usepackage[english,german]{babel}
%\usepackage[german]{babel}
%\usepackage[fixlanguage]{babelbib}


\usepackage{bbold}
\usepackage{amsthm}
\usepackage{amsmath}
\usepackage[dvipsnames]{xcolor}
\usepackage{standalone}
\usepackage{tikz}
\usepackage{cite}
\usepackage{xspace}
\usepackage{relsize}
\usepackage{mathtools} %mathclap
%\usepackage{xparse} % \newDocumentCommand for multiple optional arguments
%\usepackage{titlecaps}


%%%%%%%%%%%%%%%%%%%%%%%%%%%%%%%%%%%% THEOREMSTYLES %%%%%%%%%%%%%%%%%%%%%%%%%%%%%%%%%%

\theoremstyle{definition}
\newtheorem{definition}{Definition}[section]

\theoremstyle{theorem}
\newtheorem{theorem}{Satz}[section]

\theoremstyle{korollary}
\newtheorem{korollary}{Korollar}[section]

\theoremstyle{definition}
\newtheorem{exmp}{Beispiel}[section]


%%%%%%%%%%%%%%%%%%%%%%%%%%%%%%%%%%% MY MACROS %%%%%%%%%%%%%%%%%%%%%%%%%%%%%%%%%%%%%%%%%
%formatting
\newcommand{\comment}[2]{{\color{#1}#2}}
\newcommand{\redcomment}[1]{{\color{red}#1}}
\newcommand{\purpcomment}[1]{{\color{pink}#1}}
\newcommand{\bluecomment}[1]{{\color{blue}#1}}
\newcommand{\mt}[1]{\ensuremath{{#1}}\xspace}
\newcommand{\mynewcommand}[2]{\newcommand{#1}{\mt{#2}}} %% currently not used becaue of ide highlighting

%own names
\newcommand{\nm}[1]{#1\xspace}
\newcommand{\viewN}{\nm{view}}
\newcommand{\viewNC}{\nm{View}}
\newcommand{\viewsN}{\nm{views}}
\newcommand{\viewsNC}{\nm{Views}}
\newcommand{\grpfctN}{\nm{grouping function}}
\newcommand{\grpfctNC}{\nm{Grouping function}}
\newcommand{\grpfctNCC}{\nm{Grouping Function}}
\newcommand{\grpfctsN}{\nm{grouping functions}}
\newcommand{\grpfctsNC}{\nm{Grouping functions}}
\newcommand{\grpfctsNCC}{\nm{Grouping Functions}}
\newcommand{\chosengraphtypeNCC}{\nm{Transition System}}
\newcommand{\chosengraphtypeNC}{\nm{Transition system}}
\newcommand{\chosengraphtypeN}{\nm{transition system}}
\newcommand{\chosengraphtypesNCC}{\nm{Transition Systems}}
\newcommand{\chosengraphtypesNC}{\nm{Transition systems}}
\newcommand{\chosengraphtypesN}{\nm{transition systems}}
\newcommand{\parllcompN}{\nm{parallel composition}}
\newcommand{\parllcompNC}{\nm{Parallel composition}}
\newcommand{\parllcompNCC}{\nm{Parallel Composition}}

\newcommand{\outactident}{\nm{OutActionsIdent}}

%names
\newcommand{\iffN}{\nm{if and only if}}
\newcommand{\tsN}{\nm{TS}}

%% outactions identical
\newcommand{\outactidentstrong}{\nm{strong}}
\newcommand{\outactidentweak}{\nm{weak}}

%Identifiers for self created definitions
\newcommand{\grpfct}{\mt{F}}
\newcommand{\view}[2][\viewid]{\mt{#1_#2}}
\newcommand{\imggrp}{\mt{\arbset}}
\newcommand{\viewppty}{\mt{\grpfct}}
\newcommand{\viewid}{\mt{\ts}}
\newcommand{\pll}{\mt{||}}
\newcommand{\remstates}{\mt{\bigcup_{\state \in \states \setminus \states_1}\{\state\}}}


%example views
\newcommand{\outact}{\mt{\overrightarrow{\action}}}
\newcommand{\outacts}{\mt{\overrightarrow{\actions}}}
\newcommand{\inact}{\mt{\overleftarrow{\action}}}
\newcommand{\inacts}{\mt{\overleftarrow{\actions(\state)}}}
\newcommand{\gfctatomicprops}{\mt{{\viewppty_\atomicprops}}}
\newcommand{\gfctinitstates}{\mt{{\viewppty_\initstates}}}
\newcommand{\gfcthasoutaction}{\mt{{\viewppty_{\exists\outact}}}}
\newcommand{\gfctminoutaction}{\mt{{\viewppty_{\numoutact\leq\outact}}}}
\newcommand{\gfctmaxoutaction}{\mt{{\viewppty_{\outact\leq\numoutact}}}}
\newcommand{\gfctspanoutaction}{\mt{{\viewppty_{\numoutactb\leq\outact\leq\numoutact}}}}
\newcommand{\gfctstrongoutactident}{\mt{{\viewppty_{\outacts(\state)_=}}}}
\newcommand{\gfctweakoutactident}{\mt{{\viewppty_{\outacts(\state)_\approx}}}}
\newcommand{\gfcthasinaction}{\mt{{\viewppty_{\exists\inact}}}}
\newcommand{\gfctmininaction}{\mt{{\viewppty_{\numinact\leq\inact}}}}
\newcommand{\gfctmaxinaction}{\mt{{\viewppty_{\inact\leq\numinact}}}}
\newcommand{\gfctspaninaction}{\mt{{\viewppty_{\numinactb\leq\inact\leq\numinact}}}}
\newcommand{\gfctstronginactident}{\mt{{\viewppty_{\inacts(\state)_=}}}}
\newcommand{\gfctweakinactident}{\mt{{\viewppty_{\intacts(\state)_\approx}}}}
\newcommand{\gfctparamvalueseq}{\mt{\viewppty_{\param = \paramval}}}
\newcommand{\gfctparamvaluesneq}{\mt{\viewppty_{\param \neq \paramval}}}
\newcommand{\gfctparamvalueseqopt}[1][\paramval]{\mt{\viewppty_{\param = #1}}}
\newcommand{\gfctparamvalident}{\mt{\viewppty_{\parameval(\state,\param)}}}


\newcommand{\viewatomicprops}{\mt{\view{\gfctatomicprops}}}
\newcommand{\viewinitstates}{\mt{\view{\gfctinitstates}}}
\newcommand{\viewhasoutaction}{\mt{\view{\gfcthasoutaction}}}
\newcommand{\viewminoutaction}{\mt{\view{\gfctminoutaction}}}
\newcommand{\viewmaxoutaction}{\mt{\view{\gfctmaxoutaction}}}
\newcommand{\viewspanoutaction}{\mt{\view{\gfctspanoutaction}}}
\newcommand{\viewstrongoutactident}{\mt{\view{\gfctstrongoutactident}}}
\newcommand{\viewweakoutactident}{\mt{\view{\gfctweakoutactident}}}
\newcommand{\viewhasinaction}{\mt{\view{\gfcthasinaction}}}
\newcommand{\viewmininaction}{\mt{\view{\gfctmininaction}}}
\newcommand{\viewmaxinaction}{\mt{\view{\gfctmaxinaction}}}
\newcommand{\viewspaninaction}{\mt{\view{\gfctspaninaction}}}
\newcommand{\viewstronginactident}{\mt{\view{\gfctstronginactident}}}
\newcommand{\viewweakinactident}{\mt{\view{\gfctweakinactident}}}
\newcommand{\viewparamvalueseq}{\mt{\view{\gfctparamvalueseq}}}
\newcommand{\viewparamvaluesneq}{\mt{\view{\gfctparamvaluesneq}}}
\newcommand{\viewparamvalident}{\mt{\view{\gfctparamvalident}}}

%%OutAct
\newcommand{\numoutact}{\mt{n}}
\newcommand{\numoutactb}{\mt{m}}
\newcommand{\numinact}{\mt{n}}
\newcommand{\numinactb}{\mt{m}}
\newcommand{\setoutact}{\mt{\actions \cup \states}}

\newcommand{\predmaxoutact}[1][\numoutact]{\mt{Q_{\outact\leq#1}(\state,\state_1, \dots, \state_{#1+1})}}
\newcommand{\predminoutact}[1][\numoutact]{\mt{Q_{#1\leq\outact}(\state,\state_1, \dots, \state_{#1})}}
\newcommand{\formoutact}[1][\state]{\mt{C_{#1,\outact}}}
\newcommand{\predmaxinact}[1][\numinact]{\mt{Q_{\inact\leq#1}(\state,\state_1, \dots, \state_{#1+1})}}
\newcommand{\predmininact}[1][\numinact]{\mt{Q_{#1\leq\inact}(\state,\state_1, \dots, \state_{#1})}}

%%Parameters
\newcommand{\params}{\mt{Par}}
\newcommand{\param}{\mt{x}}
\newcommand{\paramval}{\mt{a}}
\newcommand{\parameval}{\mt{ParEval}}
\newcommand{\paramevalimg}{\mt{\arbset}}
\newcommand{\someparam}{\mt{\tilde{x}}}

%Transitionsystem
\newcommand{\ts}{\mt{TS}}
\newcommand{\state}{\mt{s}}
\newcommand{\action}{\mt{\alpha}}
\newcommand{\actionb}{\mt{\beta}}
\newcommand{\actionc}{\mt{\alpha}}
\newcommand{\atomicprop}{\mt{ap}}

\newcommand{\smstate}{\mt{\tilde{\state}}}

\newcommand{\states}{\mt{S}}
\newcommand{\actions}{\mt{Act}}
\newcommand{\transitionrel}{\mt{\longrightarrow}}
\newcommand{\initstates}{\mt{I}}
\newcommand{\atomicprops}{\mt{AP}}
\newcommand{\labelingfct}{\mt{L}}
\newcommand{\transitionsystem}{\mt
	{(\states, \actions, \transitionrel, \initstates, \atomicprops, \labelingfct)}
}
\newcommand{\eqrelview}{\mt{R}}
\newcommand{\eqclassv}[1]{\mt{\eqclass{#1}{\eqrelview}}}

%Markov chains and MDP
\newcommand{\autm}{\mt{\mathcal{M}}}
\newcommand{\probtfunc}{\mt{\textbf{P}}}
\newcommand{\initdist}{\redcomment{\mt{l_{init}}}}



%maths
\newcommand{\powerset}[1]{\mt{\mathcal{P}(#1)}}
\newcommand{\eqclass}[2]{\mt{[#1]_{#2}}}%{\mt{#1 / #2}}
\newcommand{\impr}{\mt{\hspace{3mm}\Rightarrow\hspace{2mm}}}
\newcommand{\impl}{\mt{\hspace{3mm}\Leftarrow\hspace{2mm}}}
\newcommand{\natnums}{\mt{\mathbb{N}}} 
\newcommand{\arbset}{\mt{M}}
\newcommand{\bigsum}[2][]{\mt{\mathlarger{\sum}_{#2}^{#1}}}
\newcommand{\bbigsum}[2][]{\mt{\mathlarger{\mathlarger{\sum}}_{#2}^{#1}}}
\newcommand{\invimage}[2]{#1^{\mt{-1}(#2)}}

%tickz
%% \definecolor{darkred}{RGB}{196, 42, 42}



\begin{document}
\section{View Implementation}
supposed to give a short overview how view are implemented from a conceptional perspective. No more than 1 to 2 pages.
\begin{itemize}
	\item implementation in web application pmc-vis that aims for ...
	\item the general project structure is ... prism model, database, java, json + roughly were clusters are lokaed in the project
	\item implementation of views rely on an internal graph structure that was necessary for the views that implement grouping based on structural properties of the MDP graph. Views based on MDP components firstly where implemented with direct accesses on the database but later on also adopted to the internal graph structure for performance reasons
	\item internal graph structure the \jgrapht library was used. It supplies graph structures as well as common algorithms performed on them. It was chosen because it is the most common, most up to date java library for graphs with the best documentation and broadest functionality. It is developed by ... has a java style documentation and is open source. The broad functionality assured that when implementing a view it is most certainly assured that if necessary a respective algorithm is available.
	\item The MDP was realized as an directed weighted pseudograph. The transition probabilities are stored as weights. A pseudograph has been chosen because it allows double edges as well as loops.
	\item in order of keeping the graph lean nodes and vertices are long values. The refer to the state id and the transition id respectively. The state id matches the one in the db whereas the transition id is newly generated because in the database a transition is a action with its probability distribution. This difference to the MDP definition is reversed with the internal graph structure for predefined algorithms to work properly on the MDP graph.
	\item \redcomment{For access to atomic propositions, actions and alike the graph maintains two hashmaps that map the id to the actual transition or state object respectively}
	\item depending on enum value according cluster is created
	\item normally each view is a dedicated java class, sometimes views are merged into one class that realizes combining behavior that otherwise would have been realized by composition (why? $\to$ was easy, allows more flexibility in functionality)
	\item when implementing a view only the grouping function is implemented. This makes sense because the process of actually generating the view is always the same as it was already notable in the definitions of a view, where every view is defined by its grouping function.
	\item the grouping function is calculated and its function values to the result table in a new column.
	\item visualization in general and also visualization after grouping was not part of the objective of the thesis and already present. As within the thesis there is one general approach how the resulting new graph is generated.
	\item every view class is registered in a public enum used to select which view is to be generated
	\item every view class is derived from general view class.
	\redcomment{\item in the implementation view still have the deprecated name cluster}
	\item every view class has some private attributes to store information like
	\item every view class has the methods
	\item the most relevant one is build cluster. It consist of 4 parts. 1. Checking if it is already build aborting if so. 2.create a new column in the database where the results of the grouping function are to be saved. 3. The actual computing part of the grouping function, where for every state an SQL Query adding the respective grouping function value is added to a list of SQL Query strings. 4. all the SQL queries that contain the insertion operation of the grouping function value being executed.
	\item We will take a look at this in the example..
	
\end{itemize}

\end{document}